\chapter{Analysis}
\index{Analysis%
@\emph{Analysis}}%

This report aims to demonstrate how certain traditionally difficult problems in web development have been addressed by Web Components\index{Web Components}.
The principles underlying Web Components and Polymer\index{Polymer} are summarized by the list in~\cref{sec:wcprinciples} taken from ~\cite{webcomponentscontributors2014}.
I have tried to make Speakur\index{Speakur} adhere to these principles both internally and externally,
and to demonstrate the application of abstraction\index{abstraction} and encapsulation\index{encapsulation} patterns to creating flexible, composable, reusable, sharable components.
My goal is to understand how the W3C Web Component\index{Web Components} initiative will impact web application engineering\index{software engineering} in the future.
By all appearances, the long term impact will be substantial.

\section{Web Components architecture}
Web Component's\index{Web Components} essential benefit is that it allows decoupling of components, 
separation of concerns, 
and the encapsulation of internal structure.
This raises bundles of discrete functionality or `components' to a higher level of abstraction and 
will help open up a new ecosystem of reusable, extendable, sharable modules.
The Web Component principles from~\cref{sec:wcprinciples} emphasize thinking small, 
breaking the problem down into manageable chunks and being adaptable, extendable and composable.
Web developers were already doing this in many cases, but without formal support from the browser.

Perhaps the most interesting piece of advice is to ``deliver the key benefit to HTML authors, not just coders''~\cite{webcomponentscontributors2014}.
It underscores that, fundamentally, 
the web consists of \textit{documents} with \textit{authors}, 
a fact that is sometimes forgotten by the myriad of JavaScript\index{JavaScript} libraries and frameworks\index{framework}.
The components that will find the most success will be those which focus first and foremost on providing convenience and benefit to HTML\index{HTML} authors and content developers, 
not just the software engineers\index{software engineering} who maintain the libraries and frameworks behind the scenes.
Speakur\index{Speakur} aims to provide as small a deployment surface as possible to web authors so they can easily add a discussion forum to their site without embedding a big blob of HTML.
The growth of social coding sites like GitHub\index{GitHub} will further encourage the sharing and reuse of Web Components designed along these principles.

\section{Writing Web Components with Polymer}
This section details some of the lessons learned developing Speakur\index{Speakur} on top of Google's\index{Google} Polymer\index{Polymer} framework.
Polymer's data bindings\index{data-bound template} are a powerful feature for driving `live' web applications.
Their utility extends well beyond user interface\index{user interface (UI)} views.
A 3-way automatic binding between the local model, the UI view, and the remote database is an extremely powerful abstraction,
even more so when combined with Firebase's\index{Firebase} 
Web\-Socket-based\index{WebSockets} event notification\index{event notification} system.
JavaScript variables can automatically update to reflect changes made by users on the other side of the world.
Of course, care must be taken to design the interface so the user doesn't accidentally make changes that are immediate and can't be undone. 
For example, Speakur's user preferences screen doesn't commit changes to the database until the user clicks the Save button.
Editing the text of an existing post also saves a copy of the previous version.

Data bindings and synchronization are discussed in~\cref{sec:sync}.
In general, two-way data bindings should only be used \textit{within} a component.
Across component boundaries, use a one-way binding to send data down to a child,
and fire events to notify containers of state changes.
DOM Mutation Observers\index{mutation observers} can be useful for making one component react to a change in another without directly tying their code together.
As the example with \tcode{lc} (locale) in~\cref{sec:i18n} showed, 
it's sometimes necessary to `shoehorn' certain variables into a Polymer binding expression 
to ensure that the expression is recalculated when that variable changes.

Component load times are currently one problem area for Web Components.
The Vulcanize\index{Vulcanize} tool offers definite benefits for load times as seen in~\cref{table:speakurloading},
but Speakur is a relatively simple application where all 20 or so components are used more or less immediately.
Vulcanization must be applied at the level of the site or app, not by the individual component authors.
Larger single page applications (SPAs) will face more complex challenges when trying to optimize load times, and performance in general,
before Web Components are fully native in browsers, especially when combining frameworks.

The \tcode{<template>}\index{<template>} element is a powerful new tool for web authors, 
but current implementations can have performance issues when rendering large lists of items, such as with a long list of comments. 
Polymer's \tcode{<core-list>} is more optimized for this use case.
Also, the \tcode{if=} attribute on nested \tcode{<template>}\index{<template>} elements should not be used in those cases where you could use the \tcode{hidden?=} attribute on the inner content instead.
There are several reasons for this, including performance and 
the fact that Polymer's handy \tcode{this.\$} \tcode{id}-to-object map is not dynamic; 
that is, element objects created conditionally in inner templates are not automatically added to \tcode{this.\$}.

I recommend avoiding the use of Polymer's\index{Polymer} inheritance unless there is a clear case for providing polymorphism-like behavior.
One reason for that is that (as of this writing) when a base class provides computed properties, 
these property definitions have to be repeated in any subclasses if any new ones are required,
which violates the ``don't repeat yourself'' (DRY)\index{don't repeat yourself (DRY)} principle.
Speakur uses Polymer inheritance in its CSS\index{CSS} theme components; 
most of the functionality is defined in the base class and the subclasses mainly provide numeric parameters for things like colors and margins.
Composition could achieve these behaviors almost as easily.

%Having multiple <core-overlay> active at the same time can cause problems. 
% I created a simple fix that adjusts the CSS \tcode{z-index} property. 
%You can find the fix here:~\cite{landers2015-d}.

\section{The future of Web Components}

Polymer\index{Polymer} and Web Components\index{Web Components} are extremely powerful tools.
They are relatively immature as a technology but quickly improving.
Those who wish to deploy a large, full-featured ``prod\-uction-quality'' web application based on these technologies may face some bumps in the road, 
at least until the standard is finalized and browser support is improved.
A project version below 1.0 is a general indicator of how ready its authors feel it is for production deployment, and Polymer's\index{Polymer} current version is at 0.5, soon to be 0.8, 
which suggests that 1.0 is not that far off.
Even the massively popular Angular\index{Angular} framework has announced that the controversial 2.0 re-write will be based on the Web Components standard,
so it appears that Web Components are here for the long haul~\cite{santiagoesteva2015}.

The Web Components\index{Web Components} initiative is much bigger than just the Polymer project.
The draft is due to be formally adopted as an official HTML5\index{HTML!HTML5} standard by the World Wide Web Consortium\index{W3C}.
Although it will still be a long time before full-speed native Web Component support is found in a majority of browsers worldwide, 
polyfill\index{polyfill} libraries can help bring these capabilities to current browsers.
However the prospect of forcing visitors to download a several hundred KB library just to use basic DOM functionality is not very appealing to most web engineers from a performance and user experience\index{user experience (UX)} perspective, 
so the lack of native browser support may be a limiting factor in Web Component adoption in the near term.

In essence, Web Component's\index{Web Components} key problem at the moment is that one has to download a large JavaScript\index{JavaScript} framework\index{framework} in order to use it, 
yet one of its main premises is that it frees you from needing a JavaScript framework.
Of course, the path out of this chicken-and-egg problem is for browser vendors to fully implement native Web Components in their products,
a process which won't be complete until well after the standard is finalized by the W3C.
Given the enormous size of the job that the polyfill library must accomplish, the results obtained so far are impressive. 
Speakur\index{Speakur} has demonstrated that it's possible to create focused, extensible, yet full-featured Web Components right now with the help of tools like Polymer\index{Polymer} and Firebase\index{Firebase}.
