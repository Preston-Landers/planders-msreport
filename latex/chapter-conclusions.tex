\chapter{Conclusions}
\index{Conclusions%
@\emph{Conclusions}}%

The Web Components\index{Web Components} initiative is an exciting new W3C browser standard for creating web content and applications.
The fundamental premise is that web content can be componentized, encapsulted\index{encapsulation}, and abstracted\index{abstraction} behind a well-defined APIs\index{API}.
Eventually this functionality will be delivered directly by the browser without needing a dedicated JavaScript\index{JavaScript} framework.
%But for the immediate future, using Web Components in your application means that your clients must download a large blob of JavaScript library code -- the polyfill\index{polyfill} --- 
%that enables Custom Elements\index{Custom Elements}, 
%shadow DOM\index{Shadow DOM}, 
%templates\index{HTML!Templates}, and 
%the component import system\index{HTML!Imports}.
But for the immediate future, using Web Components in your application means that your clients must download a large blob of JavaScript library code -- the polyfill\index{polyfill} --- 
that enables its constituent features: 
Custom Elements\index{Custom Elements}, 
shadow DOM\index{Shadow DOM}, 
templates\index{HTML!Templates}, and 
the component import system\index{HTML!Imports}.


The Web Components initiative has set out on the right path to becoming a mature, widely deployed technology. 
New frameworks like Polymer\index{Polymer} provide a smoother developer experience but are themselves still under preliminary development.
Established frameworks like Angular\index{Angular} have already begun the process of adapting to the new Web Component world. 
Success will be found at the end of this road as long as framework and component authors are mindful of the guiding principles of Web Components: 
to be small, focused, extensible, adaptive, 
and most of all, 
to deliver the key benefits 
not only to the software engineers\index{software engineering} who make the web run,
but to the content \textit{authors} that have made it the most important communications medium in human history.
