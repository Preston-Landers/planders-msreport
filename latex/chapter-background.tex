\chapter{Background}
\index{Background%
@\emph{Background}}%

When the Web was first created by Tim Berners-Lee in 1989, web pages were largely envisioned as static \textit{documents} with a single author or a small group of coordinating authors. 
The idea of composing a complex web application out of simple components like snapping together Lego blocks seemed like a distant dream at best.
Until recently, web authors were limited to using the predefined list of HTML elements or `tags` that were listed in the W3C standard and understood by browser programs. 
Creating your own \textit{sui generis} HTML elements with customized behaviors seemed beyond the capabilities of the web browsers of the day like Mosaic and Netscape Navigator.

As of early 2015, `modern' web apps are typically written with a Javascript framework that provides a cohesive set of structures, design patterns and practices designed to facility composing web applications, large or small, from a number of sub-components.
The difference between a `framework` and a library is somewhat arbitrary but typically frameworks are more comprehensive than a narrowly focused utility library.

\section{Current challenges in web authoring}
\subsection{Encapsulation and composition}

\section{Web Components}
\subsection{Custom HTML elements}
\subsection{Shadow DOM}
\subsection{HTML Imports}
\subsection{Templates}
\subsection{Related technologies}

\section{Literature Review}
\subsection{Popular Javascript frameworks}
\subsection{Google Polymer framework}

\section{Speakur}
\subsection{Origin}
\subsection{Motivations}

