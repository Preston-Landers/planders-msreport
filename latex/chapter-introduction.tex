\chapter{Introduction to Speakur and Web Components}
\index{Introduction to Speakur and Web Components@\emph{Introduction to Speakur and Web Components}}%
\label{ch:intro}

%\section{blah}
%\index{blah@\emph{blah}}%
%
%\texttt{asdf}
%
%\begin{quote}
%\index{guarantee}%
%This template package is provided and licensed ``as is'' without warranty
%of any kind, either expressed or implied, including, but not limited to,
%the implied warranties of merchantability and fitness for a particular
%purpose. Yadda, yadda, yadda, \ldots
%\end{quote}

\section{Software Engineering for the Web}
This report is a case study of using W3C Web Components, 
\index{Web Components}
a proposed HTML5 extension,
to implement techniques of encapsulation\index{encapsulation}, abstraction\index{abstraction}, 
and modularity in web application engineering.
I created Speakur\index{Speakur}, a real-time discussion social plugin for the web, 
as an experiment to determine the viability and maturity of using Web Components to create modern, highly composable web applications. 
Figure~\ref{f:demo1} on~\cpageref{f:demo1} shows an example Speakur discussion.

\subsection{The HTML Document Object Model}
Like most web applications, Speakur is written in a combination of HTML markup and the Java\-Script programming language. 
HTML is a declarative markup language used to create documents---web pages---which are viewed with the help of a program called the browser. 
The Hyper\-text Markup Language (HTML) 
\index{HTML|textbf}
standard has proven wildly successful since its introduction in 1993 by British physicist and computer scientist 
Tim Berners-Lee, 
\index{Berners-Lee, Tim}
with billions and billions of pages served, 
and hundreds of millions of public and private web sites forming a major part of our information landscape~\cite{internetlivestats.com2015}.
More than perhaps any other invention, the World Wide Web (WWW) has shaped how we see and use the global network.

Those designing and programming applications for the Web as a computing platform have long dreamed of the ability to mix and match independent, reusable chunks of functionality---components---in their documents without mutual coupling and interference. 
The Document Object Model (DOM)
\index{DOM}
browser abstraction does not allow for significant decoupling; 
everything lives together on one big page. Hacks like the 
\tcode{<iframe>} 
\index{<iframe>}
tag let you work around some of these limitations, usually in a inelegant and sometimes insecure fashion.

\subsection{A Programming Language for the Web}
At the time of HTML's introduction, the concept of quickly and easily composing a static web page, 
much less a full-fledged dynamic application, 
out of Lego-like reusable building blocks seemed like a distant dream at best. 
The introduction of the 
JavaScript\index{JavaScript|textbf}\footnote{JavaScript, also rendered as Javascript or JS, 
has no significant relationship to Sun's (now Oracle's) popular Java\index{Java} programming language. JavaScript is formally standardized as ECMAScript (ES)\index{ECMAScript (JavaScript)}.}
(JS) programming language to web browsers in 1995 allowed for a completely new dimension of dynamic behavior that was not previously possible~\cite{w3ccontributors2012}.
Eventually web apps like Gmail\index{GMail} and Google Docs\index{Google!Google Docs}, powered by JavaScript, rivaled traditional desktop applications in functionality and usability while being instantly accessible from anywhere with an Internet connection.
Still, web apps had to be stitched together `by hand' in ways that carefully ensured the different parts did not step on each other's toes, else disaster frequently ensued. 
Each component or sub-system could not help but be coupled to the others at some level as a result of the programming model imposed by the DOM\index{DOM} and HTML~\cite{ihrig2012}.

\subsubsection{The Growth of JavaScript}
Over the years the dynamic behaviors afforded by JavaScript grew in importance along with the web, and helped contribute to its success. 
JavaScript\index{JavaScript|textbf} is now a ubiquitous programming language and has expanded far beyond the desktop web client. 
JavaScript interpreters can be found in web servers, mobile phones, industrial controllers, and now the embedded devices that comprise the Internet of Things (IoT)\index{Internet of Things (IoT)}~\cite{flaki2015}.
Its flexible, loosely typed nature can be a boon to the prototyping and initial development process,
but the difficulties in building a large application in the DOM soon became apparent.
These problems will be explored in detail in \cref{ch:background}.

Over time, a bewildering array of frameworks and libraries sprang up around the HTML/JS ecosystem to help manage this complexity and to provide scaffolding and structure for the client side of web apps.
For many years individual JS frameworks seemed to come and go as ephemerally as teenage pop idols~\cite{allenpike2015}. 
Interoperability between these completing frameworks was virtually non-existent; 
typically, components from one framework couldn't easily be combined with those from an another.
The industry kept searching for the Next Big Thing that would make writing high quality web apps less of bug-ridden, messy chore~\cite{allenpike2015}. 

\subsubsection{The Rise of Angular}
In recent years (roughly 2011 to 2014) Google's Angular~\cite{googledevelopers2015-b}
\index{Angular|textbf}
has emerged as a dominant client framework, 
due in part to its perceived high quality and to the fact that it represents a common point for a fragmented industry to rally around~\cite{dickey2014}.
Facebook's React\index{React} library with its Virtual DOM is an up-and-comer focused on high performance that is more complementary in nature to Angular than a true challenger, 
focusing primarily on the ``view'' part of the common 
model-view-controller\index{model-view-controller} (MVC) 
design pattern~\cite{reactcontributors2015}.

Yet despite the emergence of the updated HTML5\index{HTML!HTML5|textbf} standard in 2011 and the recent successes of web frameworks like Angular\index{Angular|textbf} and React\index{React} in capturing developer attention, 
a clear picture still did not exist of how web apps could achieve the encapsulated component model that had become prevalent in other areas of software engineering\index{software engineering}.
That is, until engineers from Google\footnote{As of March 2015, Google's Chrome\index{Google!Google Chrome} browser is the most popular desktop browser~\cite{zachte2015}.}
\index{Google}
and Mozilla\footnote{The Mozilla Foundation is the sponsor of the popular Firefox\index{Mozilla!Mozilla Firefox} web browser. It grew out of Netscape, whose Navigator browser helped bring the web to a mass audience.}
\index{Mozilla} 
and other organizations got together in 2012 under a W3C Web Applications Working Group\index{Web Applications Working Group}~\cite{w3c2015} 
to draft a new standard called \textbf{Web Components} that will extend and enhance HTML5 in ways that could have a significant long-term impact~\cite{yveslafon2015}. 
For example, as of the time of this writing, a rewrite of Angular\index{Angular|textbf} called Angular 2.0 is slated to include Web Components as a core architectural element~\cite{santiagoesteva2015}.

\section{Web Components Overview}
Fundamentally, the Web Components standard consists of four new core DOM technologies---extensions to the current HTML5\index{HTML!HTML5} standard.
If these standards are accepted by major browser vendors and the World Wide Web Consortium (W3C)
\index{W3C}
which maintains HTML, 
they will eventually become native browser features and available directly to any web page without needing to use any additional JS frameworks or libraries. 
The core Web Component technologies are~\cite{penades2015}:
\begin{itemize}
\item
\textbf{Custom Elements}\index{Custom Elements}: extending HTML with author-created tags
\item
\textbf{Shadow DOM}\index{Shadow DOM}: a separate document tree completely internal to an element
\item
\textbf{Templates}\index{HTML!Templates}: scaffolding for instantiating blocks of HTML from inert templates
\item
\textbf{Imports}\index{HTML!Imports}: packaging for HTML components
\end{itemize}

\subsection{Related Technologies}
This report also explores several related web standards initiatives that are frequently associated with Web Components\index{Web Components} 
but are not formally grouped under them, including mutation observers,
\index{mutation observers}
model driven views, 
\index{Model-Driven View}
and the CSS Flexible Boxes
\index{Flex} 
and CSS Grid
\index{Grid}
systems~\cite{w3ccontributors2014,w3ccontributors2015-d,mozillacontributors2015}. 
In part because many of these technologies are not yet formally accepted as W3C standards and are not yet widely implemented in typical mobile and desktop browsers, 
I implemented Speakur\index{Speakur} using Google's experimental Polymer\index{Polymer} framework~\cite{polymercontributors2015}.
Polymer provides a JavaScript `polyfill'
\index{polyfill}
library to implement many of the new Web Component features in browsers which would otherwise not support them, 
similar to the adapter design pattern\index{design patterns}. 
Eventually this platform polyfill should become unnecessary, in theory, as WC becomes widely adopted in browsers.
Some browsers like Google Chrome 
\index{Google!Google Chrome}
have already implemented at least some native 
Web Component\index{Web Components} support 
and the polyfill\index{polyfill} is effectively a `no-op' in these areas.

\subsection{The Componentization of the Web}
The move to a component-based DOM architecture is one of the most exciting developments in web engineering in years and follows the overall growth in software-as-a-service (SaaS) 
\index{Software as a Service (SaaS)}
and the service oriented architecture
\index{Service Oriented Architecture}
model. 
The conversion of dynamic web logic---not mere snippets of plain HTML\index{HTML}---into bundles of reusable, extendable, composable services (components) enables web developers to move to a higher level of abstraction\index{abstraction} than was previously possible.

The shift towards a component-based Web will enable interesting new composite services, mashups, and may help broaden the potential pool of web developers. 
What previously required a highly integrated, high-overhead development model or lots of tedious glue code can become as simple as importing a custom element and dropping it onto a page.
Integrated development environments (IDEs) and other tools that embrace Web Components will be able to assist content authors by, for example, offering name completion on custom elements and the attributes that form their API\index{API}.

\section{Structure of This Report}
\index{Structure of This Report@\emph{Structure of This Report}}%

The goal of this report is to demonstrate the application of 
software engineering\index{software engineering} 
design patterns\index{design patterns} embodied in the W3C proposed Web Components standard such as 
encapsulating\index{encapsulation} internal logic 
behind abstraction\index{abstraction} layers, 
modular composition, 
and the real-time synchronization of web application state. 
This report discusses many of the goals and principles of the Web Components initiative and how a number of different technologies taken together help raise the overall level of 
abstraction\index{abstraction} for content authors, web engineers, and application developers---which I will refer to collectively as (web) authors for short.

\Cref{ch:background} gives an introduction to some of the architectural problems inherent in modern web authoring and how Web Components\index{Web Components} (WC) address them. 
It provides background on the software engineering design patterns that are embodied in Web Components like encapsulation and composition.
It describes some of the motivations behind the development of Speakur\index{Speakur} and some of the specific software engineering design questions it addresses.

\Cref{ch:approach} describes the high level software architecture\index{architecture} choices behind Speakur and details the specific structures and techniques used when constructing Web Components\index{Web Components}.
It describes how Speakur\index{Speakur} uses WC to implement encapsulated\index{encapsulation} modules whose internals are protected from unintentional outside influence. 
It also describes how the choice of the Firebase\index{Firebase} cloud database service, its WebSocket-based\index{WebSockets} event notification system, and its security subsystem have impacted Speakur's architecture.

In \cref{ch:implementation}, Implementing a Web Component, I show how applying Web Component\index{Web Components} principles eases the task of creating a flexible discussion forum for both desktop and mobile browsers. 
I describe the low level architecture\index{architecture}, code flow, and data synchronization process, 
as well as how to internationalize\index{internationalization} a JavaScript application so the user interface language (locale\index{localization}) can be switched on the fly.
Another important topic in this section is security\index{security}: how can we implement a largely client-based system while maintaining some kind of data integrity?

This is followed by~\cref{ch:analysis}, Analysis and Lessons Learned, which discusses some of the outcomes 
as compared to the original goals and also looks at the impact of the selection of Web Components\index{Web Components}, 
Polymer\index{Polymer}, Firebase\index{Firebase} and several other architectural\index{architecture} choices. 
Finally, the Conclusion summarizes how the Web Components\index{Web Components} initiative will positively impact the future of software engineering\index{software engineering} for the web. 

\section{Source Code and Demonstration Resources}
\index{Speakur!source code}

The source code for Speakur\index{Speakur} consists of HTML\index{HTML} and JavaScript\index{JavaScript} files located in a Git\index{Git} version control repository. 
These files constitute an \textit{HTML Import}\index{HTML!Imports} package that provides a
\textbf{\tcode{<speakur-discussion>}}
custom HTML element\index{Custom Elements} for the use of web authors in their own pages.
An example of using \tcode{<speakur-discussion>} is given in Chapter 3 and in briefly in Listing~\ref{l:example1} below.

%\pagebreak
\begin{lstlisting}[language=HTML5,caption={Speakur custom HTML element.},label=l:example1]
 <!-- place this on your page
      where you want a discussion -->
 <speakur-discussion
   href="http://example.com/news/web-components-in-action"
   allowAnonymous="false">
 </speakur-discussion>
\end{lstlisting}

The Speakur\index{Speakur} source code and component documentation can be found on the social coding site GitHub.com\index{GitHub} at~\cite{landers2015-b}:

\tcode{\url{https://github.com/Preston-Landers/speakur-discussion}}

A demonstration web page that shows off a live embedded Speakur discussion is available at the following location~\cite{landers2015-c}:

\tcode{\url{https://preston-landers.github.io/speakur-discussion/components/speakur-discussion/demo.html}}
