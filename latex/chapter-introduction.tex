\chapter{Introduction}
\index{Introduction@\emph{Introduction}}%

%\section{blah}
%\index{blah@\emph{blah}}%
%
%\texttt{asdf}
%
%\begin{quote}
%\index{guarantee}%
%This template package is provided and licensed ``as is'' without warranty
%of any kind, either expressed or implied, including, but not limited to,
%the implied warranties of merchantability and fitness for a particular
%purpose. Yadda, yadda, yadda, \ldots
%\end{quote}

This report describes Speakur, a real time discussion system for the mobile web, and 
presents a case study of applying W3C Web Components 
\index{Web Components}
to achieve encapsulation and separation of concerns within the context of 
collaborative web authoring. 

The potential componentization of web is one of the most exciting innovations and mirrors the overall growth in the Service Oriented Architecture 
\index{Service Oriented Architecture}
as a organizational and deployment concept. 
The conversions of dynamic web logic, not just 'dead' snippets of HTML, into bundles of reusable, extendable, composable components enables web developers to move to a higher level of abstraction than was previously possible.

The componentization of the web will enable all sorts of interesting new composite services, mashups, and help broaden the potential pool of web developers. 
It will do this by allowing authors to publish easily reusable 'widgets' that can be easily juggled around and combined in novel ways that previously required a highly integrated (and hugely expensive) development model.

\section{Structure of This Paper}
\index{Structure of This Paper@\emph{Structure of This Paper}}%

The goal of this paper is to demonstrate the application of software engineering design patterns embodied in the  W3C proposed Web Components standard such as encapsulation, composition, and
automatic synchronization of application state. 
This paper attempts to explain many of the goals and principles of the Web Components initiative and show how a number of different technologies taken together help raise the overall level of abstraction for web authors and developers.


The Background section of this report provides an introduction some of the architectural problems inherent in modern web authoring and how Web Components (WC) attempts to address these issues.
It also describes the specific software engineering problem that Speakur is attempting to solve, which is the ability to provide a hassle-free way to host an embedded discussion forum inside an arbitrary web resource in a way that is fully encapsulated.

The Approach section details the specific structures and techniques used when constructing a Web Component. 


\section{Source Code and Demonstration Resources}
\index{Source Code and Demonstration Resources@\emph{Source Code and Demonstration Resources}}%

The source code for Speakur consists of HTML and Javascript code located at the following git repository:

The public documentation for web authors to use Speakur in their own sites can be found here:

Demonstrations of several web pages which show off embedded Speakur discussions are available at the following location:

