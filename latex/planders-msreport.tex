% Preston Landers  <planders@utexas.edu>
% Master's Report document 
%
% Major: Computer and Electrical Engineering
% Focus: Software Engineering
% Cockrell School of Engineering
% The University of Texas at Austin
%
% Based on the UT LaTeX package provided at:
% http://www.utexas.edu/ogs/etd/LaTeX/



\documentclass[12pt]{report}	% The documentclass must be ``report''.

\usepackage{utdiss2}  		% Dissertation package style file.

%\usepackage[driver=pdftex,margin=1.2in,letterpaper]{geometry}
%\usepackage[driver=xetex,margin=1.2in,letterpaper]{geometry}

\usepackage{geometry}

\newenvironment{changemargin}[2]{%
\begin{list}{}{%
\setlength{\topsep}{0pt}%
\setlength{\leftmargin}{#1}%
\setlength{\rightmargin}{#2}%
\setlength{\listparindent}{\parindent}%
\setlength{\itemindent}{\parindent}%
\setlength{\parsep}{\parskip}%
}%
\item[]}{\end{list}}

%%%%%%%%%%%%%%%%%%%%%%%%%%%%%%%%%%%%%%%%%%%%%%%%%%%%%%%%%%%%%%%%%%%%%%
% Optional packages used for this sample dissertation. If you don't  %
% need a capability in your dissertation, feel free to comment out   %
% the package usage command.					     %
%%%%%%%%%%%%%%%%%%%%%%%%%%%%%%%%%%%%%%%%%%%%%%%%%%%%%%%%%%%%%%%%%%%%%%

\usepackage{amsmath,amsthm,amsfonts,amscd} 
				% Some packages to write mathematics.
\usepackage{eucal} 	 	% Euler fonts
\usepackage{verbatim}      	% Allows quoting source with commands.
\usepackage{makeidx}       	% Package to make an index.
\usepackage{graphicx}

\usepackage[hyphens]{url}		% Allows good typesetting of web URLs.

\usepackage{booktabs}  % table formatting
\newcommand{\ra}[1]{\renewcommand{\arraystretch}{#1}}
%\usepackage{float}
%\restylefloat{table}

\usepackage{caption}
\usepackage{subcaption}  % for sub-figures

\usepackage[backend=biber,url=false,citestyle=numeric,firstinits=false]{biblatex}
\usepackage{citesort}         	% 

\addbibresource{diss.bib}

% Prestons code listing definitions
\input{codelistings}

%\usepackage{xunicode}
%\usepackage[T1]{fontenc}
\usepackage{fontspec}
\usepackage{fixltx2e} % fix a problem with fnsymbol w/ fontspec, such as footnotes in vita

\usepackage[htt]{hyphenat}

% NOTE: technically the report formatting requirements say only
% a single font must be used. But I've seen many examples of 
% UT master's reports that use a monospace/teletype face for 
% technical terms. I believe that's considered a 'math font' so it's ok...

%\setmonofont[Scale=0.95]{Inconsolata}
%\setmonofont{Inconsolata}
%\usepackage{luximono}
%\usepackage[tt={oldstyle=true,variable=true}]{cfr-lm}
%\renewcommand{\ttdefault}{clmjv}

% My version of Inconsolata may not support bold/italic?
%\setmonofont[SmallCapsFont={Latin Modern Mono Caps}]{Latin Modern Mono Light}

\setmainfont[Ligatures=TeX]{Adobe Text Pro}
%\setmainfont[Ligatures=TeX]{Minion Pro}
\setmonofont[Scale=0.95]{Consolas}

%\usepackage{libertine}%% Only as example for the romans/sans fonts
%\usepackage{beraserif}
%\usepackage[scaled=0.9]{beramono}

\usepackage{hyperref}  % adds PDF bookmarks and such
\hypersetup{
 colorlinks=true,
 linkcolor=black,  % blue
 citecolor=black,  % green
 urlcolor=black,
% pdffitwindow=true,
 pdftitle={Speakur: Leveraging Web Components for Composable Applications},
 pdfauthor={Preston Landers <planders@utexas.edu>},
 pdfsubject={A case study in HTML5 Web Components, the Polymer framework, and Firebase.},
 pdfkeywords={Polymer} {Web Components} {HTML5} {Speakur} {Firebase},
 }
\hypersetup{breaklinks=true}
\usepackage{ragged2e}

\usepackage[capitalize]{cleveref}

% Force cleveref ``page'' references to be lowercase despite the capitalize option above
\crefformat{page}{page~#2#1#3}

%\pretolerance=10000

%\linespread{1.5} % Line spacing

%\usepackage{draftcopy}		% Uncomment this line to have the
				% word, "DRAFT," as a background
				% "watermark" on all of the pages of
				% of your draft versions. When ready
				% to generate your final copy, re-comment
				% it out with a percent sign to remove
				% the word draft before you re-run
				% Makediss for the last time.

\author{Preston Brent Landers}  	% Required

\address{\texttt{planders@utexas.edu} \\~\\ \url{https://github.com/Preston-Landers/} }  % Required

% Required
\title{Speakur: Leveraging Web Components \\ for Composable Applications}

%%%%%%%%%%%%%%%%%%%%%%%%%%%%%%%%%%%%%%%%%%%%%%%%%%%%%%%%%%%%%%%%%%%%%%
% NOTICE: The total number of supervisors and other members %%%%%%%%%%
%%%%%%%%%%%%%%% MUST be seven (7) or less! If you put in more, %%%%%%%
%%%%%%%%%%%%%%% they are put on the page after the Committee %%%%%%%%%
%%%%%%%%%%%%%%% Certification of Approved Version page. %%%%%%%%%%%%%%
%%%%%%%%%%%%%%%%%%%%%%%%%%%%%%%%%%%%%%%%%%%%%%%%%%%%%%%%%%%%%%%%%%%%%%

%%%%%%%%%%%%%%%%%%%%%%%%%%%%%%%%%%%%%%%%%%%%%%%%%%%%%%%%%%%%%%%%%%%%%%
%
% Enter names of the supervisor and co-supervisor(s), if any,
% of your dissertation committee. Put one name per line with
% the name in square brackets. The name on the last line, however,
% must be in curly braces.
%
% If you have only one supervisor, the entry below will read:
%
%	\supervisor
%		{Supervisor's Name}
%
% NOTE: Maximum three supervisors. Minimum one supervisor.
% NOTE: The Office of Graduate Studies will accept only two supervisors!
% 
%
%\supervisor
%	[Adnan Aziz]
%	{Johannes Kepler}
\supervisor
	{Adnan Aziz}

%%%%%%%%%%%%%%%%%%%%%%%%%%%%%%%%%%%%%%%%%%%%%%%%%%%%%%%%%%%%%%%%%%%%%%
%
% Enter names of the other (non-supervisor) members(s) of your
% dissertation committee. Put one name per line with the name
% in square brackets. The name on the last line, however, must
% be in curly braces.
%
% NOTE: Maximum six other members. Minimum zero other members.
% NOTE: The Office of Graduate Studies may restrict you to a total
%	of six committee members.
%
%
%\committeemembers
%	[Erwin Schr\"odinger]
%	[Albert Einstein]
%	[Charles Townes]
%	{Arthur Schawlow}
\committeemembers
	{Christine Julien}


%%%%%%%%%%%%%%%%%%%%%%%%%%%%%%%%%%%%%%%%%%%%%%%%%%%%%%%%%%%%%%%%%%%%%%

\previousdegrees{B.A.}
     % The abbreviated form of your previous degree(s).
     % E.g., \previousdegrees{B.S., MBA}.
     %
     % The default value is `B.S., M.S.'

\graduationmonth{May}
     % Graduation month, either May, August, or December, in the form
     % as `\graduationmonth{May}'. Do not abbreviate.
     %
     % The default value (either May, August, or December) is guessed
     % according to the time of running LaTeX.

\graduationyear{2015}
     % Graduation year, in the form as `\graduationyear{2001}'.
     % Use a 4 digit (not a 2 digit) number.
     %
     % The default value is guessed according to the time of 
     % running LaTeX.

%\typist{...}       
     % The name(s) of typist(s), put `the author' if you do it yourself.
     % E.g., `\typist{Maryann Hersey and the author}'.
     %
     % The default value is `the author'.


%%%%%%%%%%%%%%%%%%%%%%%%%%%%%%%%%%%%%%%%%%%%%%%%%%%%%%%%%%%%%%%%%%%%%%
% Commands for master's theses and reports.			     %
%%%%%%%%%%%%%%%%%%%%%%%%%%%%%%%%%%%%%%%%%%%%%%%%%%%%%%%%%%%%%%%%%%%%%%
%
% If the degree you're seeking is NOT Doctor of Philosophy, uncomment
% (remove the % in front of) the following two command lines (the ones
% that have the \ as their second character).
%
\degree{Master of Science in Engineering}
\degreeabbr{M.S.E.}

% Uncomment the line below that corresponds to the type of master's
% document you are writing.
%
\masterreport
%\masterthesis


%%%%%%%%%%%%%%%%%%%%%%%%%%%%%%%%%%%%%%%%%%%%%%%%%%%%%%%%%%%%%%%%%%%%%%
% Some optional commands to change the document's defaults.	     %
%%%%%%%%%%%%%%%%%%%%%%%%%%%%%%%%%%%%%%%%%%%%%%%%%%%%%%%%%%%%%%%%%%%%%%
%
%\singlespacing
%\oneandonehalfspacing

%\singlespacequote
\oneandonehalfspacequote

\topmargin 0.125in	% Adjust this value if the PostScript file output
			% of your dissertation has incorrect top and 
			% bottom margins. Print a copy of at least one
			% full page of your dissertation (not the first
			% page of a chapter) and measure the top and
			% bottom margins with a ruler. You must have
			% a top margin of 1.5" and a bottom margin of
			% at least 1.25". The page numbers must be at
			% least 1.00" from the bottom of the page.
			% If the margins are not correct, adjust this
			% value accordingly and re-compile and print again.
			%
			% The default value is 0.125"

		% If you want to adjust other margins, they are in the
		% utdiss2-nn.sty file near the top. If you are using
		% the shell script Makediss on a Unix/Linux system, make
		% your changes in the utdiss2-nn.sty file instead of
		% utdiss2.sty because Makediss will overwrite any changes
		% made to utdiss2.sty.

%%%%%%%%%%%%%%%%%%%%%%%%%%%%%%%%%%%%%%%%%%%%%%%%%%%%%%%%%%%%%%%%%%%%%%
% Some optional commands to be tested.				     %
%%%%%%%%%%%%%%%%%%%%%%%%%%%%%%%%%%%%%%%%%%%%%%%%%%%%%%%%%%%%%%%%%%%%%%

% If there are 10 or more sections, 10 or more subsections for a section,
% etc., you need to make an adjustment to the Table of Contents with the
% command \longtocentry.
%
%\longtocentry 



%%%%%%%%%%%%%%%%%%%%%%%%%%%%%%%%%%%%%%%%%%%%%%%%%%%%%%%%%%%%%%%%%%%%%%
%	Some math support.					     %
%%%%%%%%%%%%%%%%%%%%%%%%%%%%%%%%%%%%%%%%%%%%%%%%%%%%%%%%%%%%%%%%%%%%%%
%
%	Theorem environments (these need the amsthm package)
%
%% \theoremstyle{plain} %% This is the default

\newtheorem{thm}{Theorem}[section]
\newtheorem{cor}[thm]{Corollary}
\newtheorem{lem}[thm]{Lemma}
\newtheorem{prop}[thm]{Proposition}
\newtheorem{ax}{Axiom}

\theoremstyle{definition}
\newtheorem{defn}{Definition}[section]

\theoremstyle{remark}
\newtheorem{rem}{Remark}[section]
\newtheorem*{notation}{Notation}

%\numberwithin{equation}{section}


%%% Preston custom macros
\newcommand{\tcode}[1]{\texttt{#1}}

%%%%%%%%%%%%%%%%%%%%%%%%%%%%%%%%%%%%%%%%%%%%%%%%%%%%%%%%%%%%%%%%%%%%%%
%	Macros.							     %
%%%%%%%%%%%%%%%%%%%%%%%%%%%%%%%%%%%%%%%%%%%%%%%%%%%%%%%%%%%%%%%%%%%%%%
%
%	Here some macros that are needed in this document:


\newcommand{\latexe}{{\LaTeX\kern.125em2%
                      \lower.5ex\hbox{$\varepsilon$}}}

\newcommand{\amslatex}{\AmS-\LaTeX{}}

\chardef\bslash=`\\	% \bslash makes a backslash (in tt fonts)
			%	p. 424, TeXbook

\newcommand{\cn}[1]{\texttt{\bslash #1}}

\makeatletter		% Starts section where @ is considered a letter
			% and thus may be used in commands.
\def\square{\RIfM@\bgroup\else$\bgroup\aftergroup$\fi
  \vcenter{\hrule\hbox{\vrule\@height.6em\kern.6em\vrule}%
                                              \hrule}\egroup}
\makeatother		% Ends sections where @ is considered a letter.
			% Now @ cannot be used in commands.

\makeindex    % Make the index


%%%%%%%%%%%%%%%%%%%%%%%%%%%%%%%%%%%%%%%%%%%%%%%%%%%%%%%%%%%%%%%%%%%%%%
%		The document starts here.			     %
%%%%%%%%%%%%%%%%%%%%%%%%%%%%%%%%%%%%%%%%%%%%%%%%%%%%%%%%%%%%%%%%%%%%%%

\begin{document}

\copyrightpage          % Produces the copyright page.


%
% NOTE: In a doctoral dissertation, the Committee Certification page
%		(with signatures) is BEFORE the Title page.
%	In a masters thesis or report, the Signature page
%		(with signatures) is AFTER the Title page.
%
%	If you are writing a masters thesis or report, you MUST REVERSE
%	the order of the \commcertpage and \titlepage commands below.
%
\commcertpage           % Produces the Committee Certification
			%   of Approved Version page (doctoral)
			%   or Signature page (masters).
			%		20 Mar 2002	cwm

\titlepage              % Produces the title page.



%%%%%%%%%%%%%%%%%%%%%%%%%%%%%%%%%%%%%%%%%%%%%%%%%%%%%%%%%%%%%%%%%%%%%%
% Dedication and/or epigraph are optional, but must occur here.      %
%%%%%%%%%%%%%%%%%%%%%%%%%%%%%%%%%%%%%%%%%%%%%%%%%%%%%%%%%%%%%%%%%%%%%%
%
\begin{dedication}
\index{Dedication@\emph{Dedication}}%
This report is dedicated to my wife for her patience,\linebreak
and to my parents for their support and guidance
\end{dedication}


\begin{acknowledgments}		% Optional
\index{Acknowledgments@\emph{Acknowledgments}}%
I would like to thank Bill Balcezak for strongly encouraging and facilitating the completion of my undergraduate and graduate education. 
I would also like to thank Curt Finch, John Maddalozzo and everyone else at Journyx for their patience, understanding and encouragement.

I would also like to thank my supervisor, Dr. Adnan Aziz, 
for his enthusiasm, invaluable feedback, wisdom, and for teaching one of best classes I've ever taken.
In addition I would like to thank Dr. Christine Julien, Mr. Bill Bard, Dr. Kathleen Barber, and all of the other excellent instructors in the CLEE program.
Without this amazing program and everyone mentioned above, I would not be who I am today.
\end{acknowledgments}


% The abstract is required. Note the use of ``utabstract'' instead of
% ``abstract''! This was necessary to fix a page numbering problem.
% The abstract heading is generated automatically.
% Do NOT use \begin{abstract} ... \end{abstract}.
%
\utabstract
\index{Abstract}%
\indent

This report is a case study of applying abstraction, encapsulation, and composition techniques to web application architecture with the use of Web Components, 
a proposed extension to the HTML5 Document Object Model. 
I created Speakur, a real-time social discussion plugin for the mobile and desktop web, 
to show how Web Components can help realize software engineering principles and design patterns, including the composition of applications from components sourced from diverse authors and frameworks.

Web authors can add a Speakur discussion to their page by inserting a simple HTML element at the desired spot to give the page a real-time discussion or feedback system.
Speakur uses the Polymer framework's implementation of the draft Web Components standard to achieve the encapsulation of its internal implementation details from the containing page behind a simplified, well defined interface.
Web Components are a proposed W3C standard for writing custom HTML tags that take advantage of new browser technologies like Shadow DOM, package importing, CSS Flexboxes and data-bound templates.

This report reviews Web Components and related technologies and provides a case study for structuring a real-world WC applet that is embedded in a larger app or system.
The major research question is whether Web Components offer a viable path towards the encapsulation and composition principles that have largely eluded web engineers thus far. 
In other words, \textit{are components really the future of the web}? 
Subsidiary topics include assessing the maturity and performance of current Web Components technologies,
and methods of synchronization between user interface components and local and remote data models.
My analysis shows that Web Components successfully address many of the key structural and organizational problems faced by web software engineers.

\tableofcontents   % Table of Contents will be automatically
                   % generated and placed here.

\listoftables      % List of Tables and List of Figures will be placed
\listoffigures     % here, if applicable.
\lstlistoflistings


%%%%%%%%%%%%%%%%%%%%%%%%%%%%%%%%%%%%%%%%%%%%%%%%%%%%%%%%%%%%%%%%%%%%%%
% Actual text starts here.					     %
%%%%%%%%%%%%%%%%%%%%%%%%%%%%%%%%%%%%%%%%%%%%%%%%%%%%%%%%%%%%%%%%%%%%%%
%
% Including external files for each chapter makes this document simpler,
% makes each chapter simpler, and allows for generating test documents
% with as few as zero chapters (by commenting out the include statements).
% This allows quicker processing by the Makediss command file in case you
% are not working on a specific, long and slow to compile chapter. You
% can even change the chapter order by merely interchanging the order
% of the include statements (something I found helpful in my own
% dissertation).
%
\chapter{Introduction}
\index{Introduction@\emph{Introduction}}%

%\section{blah}
%\index{blah@\emph{blah}}%
%
%\texttt{asdf}
%
%\begin{quote}
%\index{guarantee}%
%This template package is provided and licensed ``as is'' without warranty
%of any kind, either expressed or implied, including, but not limited to,
%the implied warranties of merchantability and fitness for a particular
%purpose. Yadda, yadda, yadda, \ldots
%\end{quote}

This report describes Speakur, a real time discussion system for the mobile web, and 
presents a case study of applying W3C Web Components 
\index{Web Components}
to achieve encapsulation and separation of concerns within the context of 
collaborative web authoring. 

The potential componentization of web is one of the most exciting innovations and mirrors the overall growth in the Service Oriented Architecture 
\index{Service Oriented Architecture}
as a organizational and deployment concept. 
The conversions of dynamic web logic, not just 'dead' snippets of HTML, into bundles of reusable, extendable, composable components enables web developers to move to a higher level of abstraction than was previously possible.

The componentization of the web will enable all sorts of interesting new composite services, mashups, and help broaden the potential pool of web developers. 
It will do this by allowing authors to publish easily reusable 'widgets' that can be easily juggled around and combined in novel ways that previously required a highly integrated (and hugely expensive) development model.

\section{Structure of This Paper}
\index{Structure of This Paper@\emph{Structure of This Paper}}%

The goal of this paper is to demonstrate the application of software engineering design patterns embodied in the  W3C proposed Web Components standard such as encapsulation, composition, and
automatic synchronization of application state. 
This paper attempts to explain many of the goals and principles of the Web Components initiative and show how a number of different technologies taken together help raise the overall level of abstraction for web authors and developers.


The Background section of this report provides an introduction some of the architectural problems inherent in modern web authoring and how Web Components (WC) attempts to address these issues.
It also describes the specific software engineering problem that Speakur is attempting to solve, which is the ability to provide a hassle-free way to host an embedded discussion forum inside an arbitrary web resource in a way that is fully encapsulated.

The Approach section details the specific structures and techniques used when constructing a Web Component. 


\section{Source Code and Demonstration Resources}
\index{Source Code and Demonstration Resources@\emph{Source Code and Demonstration Resources}}%

The source code for Speakur consists of HTML and Javascript code located at the following git repository:

The public documentation for web authors to use Speakur in their own sites can be found here:

Demonstrations of several web pages which show off embedded Speakur discussions are available at the following location:



\chapter{Background}
\index{Background%
@\emph{Background}}%

When the Web was first created by Tim Berners-Lee in 1989, web pages were largely envisioned as static \textit{documents} with a single author or a small group of coordinating authors. 
The idea of composing a complex web application out of simple components like snapping together Lego blocks seemed like a distant dream at best.
Until recently, web authors were limited to using the predefined HTML layout elements or `tags' that were listed in the W3C standard and understood by browser programs, such as \tcode{<title>} and \tcode{<video>}. 
Creating your own \textit{sui generis} HTML elements with unique behaviors seemed beyond the capabilities of the web browsers of the day like Mosaic and Netscape Navigator.

As of early 2015, modern web apps are typically written with a Javascript framework that provides a cohesive set of structures, design patterns and practices designed to facilitate composing web applications, large or small, from a number of sub-components.
Angular, Meteor, and Backbone are three such frameworks.
The difference between a `framework' and a library is somewhat arbitrary, but typically frameworks are more comprehensive than narrowly focused utility libraries.
Yet all frameworks must exist within the confines of the programming model provided by the browser and the Document Object Model (DOM). 
In this model, the entire web page or app belongs to a single `document', constituent parts are not encapsulated or isolated from each other, and authors are limited to working with the predefined HTML tags.
These issues make it difficult to create and share generic, reusable \textit{web components} 
--- in the abstract sense --- 
among different users who may not use the same frameworks or follow the same set of assumptions and conventions.

\section{Current challenges in web authoring}
In object oriented programming, encapsulation is typically defined as a 
``language mechanism for restricting access to some of the object's components''
~\cite[p. 522]{mitchell2003}.
The point of encapsulation is providing an \textit{abstraction} that consumers of the functionality can rely on without knowing internals. 
The goals of encapsulation and abstraction include:
\begin{quote}
Identifying the interface of a data structure \dots 
providing \textit{information hiding} by separating implementation decisions from parts of the program that use the data structure \dots 
and allowing the data structure to be used in many different ways by many different components.
\cite[p. 243]{mitchell2003}
\end{quote}

Although techniques of abstraction and encapsulation have been widespread in object oriented programming for decades,
the fundamental web client programming model has not allowed for significant encapsulation of things like the DOM structure and CSS style rules~\cite{ihrig2012}.

To illustrate how these problems affect the ability of authors to share and reuse code, let's look at an example from the popular Twitter Bootstrap library~\cite{bootstrapcontributors2015}.
Twitter Bootstrap is a collection of Cascading Style Sheet (CSS) rules and Javascript widgets or components designed to allow web authors to quickly `bootstrap' an attractive, consistent look-and-feel onto a web page.
Bootstrap provides pre-styled User Interface (UI) widgets such as menus, buttons, panels, dropdown selectors, alerts, dialogs, and so on, to be used as building blocks to construct web sites or application user interfaces.
Because Bootstrap must work within the confines of the DOM and the HTML5 standard, this necessarily exposes a great deal of Bootstrap's internals to its users.
For example, to add a Bootstrap site navigation bar to your page, you must essentially copy and paste a large block of HTML and then customize it to your needs as shown in figure~\ref{f:twbs1}.

% 
\begin{figure}[htb]
\centering
 \includegraphics[width=6in]{images/bootstrap_navbar_html.png}
\caption{A partial example of Twitter Bootstrap navigation bar HTML.}
\label{f:twbs1}
\end{figure}
\index{commands!environments!figure}%

This forces Bootstrap's users to tightly couple the layout of their page with the internal structure required by Bootstrap's navigation bar widget. 
This coupling militates against Bootstrap significantly refactoring the internal structure of the navigation widget because that would require a large community of developers to update their applications accordingly.
In addition, because CSS rules normally apply across the entire page, the authors of Bootstrap must carefully select the scope and nomenclature of all rules to ensure minimal interference with other components and unintended effects. 
Even then, conflicts are inevitable when the entire page is treated as a single sandbox and you combine components from many different vendors. 

What if instead one could create and share a reusable chunk of functionality --- a web component -- that hid all of these tedious structural details and encapsulated its private, internal state? 
What if web authors could create their \textit{own} HTML elements?  
Using Bootstrap's navigation bar could be as simple as replacing the code in figure~\ref{f:twbs1} with a custom element like the one in 
% figure~\ref{f:twbs2}.
%listing~\ref{l:twbs2}.
the following example:

% 
%\begin{figure}[htb]
%\centering
% \includegraphics[width=3.5in]{images/bootstrap_navbar_wc.png}
%\caption{Hypothetical Bootstrap nav bar custom element.}
%\label{f:twbs2}
%\end{figure}
%\index{commands!environments!figure}%

% TODO: replace that with a regular listing?

\begin{lstlisting}[language=HTML5,numbers=none,caption=
{Hypothetical Bootstrap nav bar custom element.},label=l:twbs2,captionpos=below]
 <twbs-navbar>
   <a href="#">Home</a>
   <a href="#">About</a>
   <a href="#">Sign In</a>
 </twbs-navbar>
\end{lstlisting}

\subsection{Encapsulation and composition}

The Web Components working group, consisting of software engineers from several major browser vendors, looked at this situation and found that, in practice, browsers already had a suitable model for encapsulating components that hide complexity behind well-defined interfaces.
That model was that one used internally by browsers to implement the newer HTML5 tags like the \textbf{\tcode{<video>}} element. 
The \tcode{<video>} element presents a simple interface (API) to HTML authors that hides the complexities of playing high definition video.
Internally, however, browsers implement \tcode{<video>} with a `shadow' or hidden document inside the object that contains the internal state. 
For example, an author can write:
\begin{lstlisting}[language=HTML5,numbers=none]
	<video loop src=...> </video>
\end{lstlisting}
to cause the video to loop repeatedly.

This shadow Document Object Model (DOM) inside the \tcode{<video>} tag creates the user interface (UI) needed to control video playback such as the volume controls, the timeline bar, and the pause and play buttons.
These inner playback controls are themselves built out of HTML, CSS and JS but these details are not exposed to web authors who simply place a \tcode{<video>} element on their page. 
Figure~\ref{f:html5video} illustrates how this works. It shows the shadow (internal) DOM of a \tcode{<video>} element on a page with the Play button \tcode{<div>} highlighted.

% 
\begin{figure}[htb]
\centering
 \includegraphics[width=5.5in]{images/html5_video_control.png}
\caption{Opera's shadow DOM for \tcode{<video>} highlighting the Play button}
\label{f:html5video}
\end{figure}
\index{commands!environments!figure}%

This example illustrates two design principles that are widely followed in other areas of software engineering [CITE]:
\begin{itemize}
\item Use \textbf{encapsulation} and well defined interfaces, as the \tcode{<video>} element does, to protect private state, hide implementation complexity, and leave implementors free to refactor internals.
\item Prefer \textbf{composition} or \textit{has-a} relationships over inheritance or \textit{is-a} relationships when building modules. 
\end{itemize}

Composition helps reduce coupling or structural between modules by forcing them to interact using only public interfaces.
In the case of the interface for \tcode{<video>}, it's composed of simple block elements and scoped CSS roles and the Volume and Play controls aren't particularly special objects, just \tcode{<divs>} with CSS rules and click handlers.

The solution, therefore, to these coupling problems in web authoring is to expose these internal brower APIs for creating elements in a safe and portable fashion. 
This will allow web authors to create their own rich custom elements using standard portable APIs, encapsulate their internals, and enable easier sharing, composition and integration.
The question remains, which specific browser internals must be exposed and standardized in order to support Web Components?

\section{Web Components}

The Web Component initiative consists of two main technologies and two supporting features. 
Custom HTML Elements and Shadow DOM are the two key players while HTML Imports and Templates support these features. One of the central goals of the Web Components initiatives is to maintain interoperability across different browsers and frameworks, so that modules which adhere to the Web Components standard can provide a consistent experience no matter what framework the developer chooses or which browser the user selects.

\subsection{Custom HTML elements}
Never before have web authors been able to define their own custom HTML elements that were not found in the official list.
Actually, many authors and web frameworks have been doing exactly that for years, primarily for internal purposes where the custom elements are pre\-processed and compiled down to standard HTML.
The custom elements would not get sent to the end user's browser because it would not know what to do with them.
Technically, the DOM has long supported creating custom-named elements, but it was not possible to do much with them because they were treated like an ordinary \tcode{<span>}.
However the possibility now exists to create custom elements in a standard way that will work consistently across browsers. 

TODO: Custom elements:

\url{http://www.w3.org/TR/custom-elements/}

The primary restriction is that all custom elements must have a \texttt{-} character (dash) in their name, such as \tcode{<my-element>}. 
This is to avoid a name collision with future built-in HTML elements. 
To create a new Custom Element, you must first register the element:

\begin{lstlisting}[language=JavaScript,numbers=none]
 var MyElement = document.registerElement('my-element');
\end{lstlisting}

Then you place your new element on the page, either declaratively in HTML:

\begin{lstlisting}[language=HTML5,numbers=none]
 <my-element> hello, world! </my-element>
\end{lstlisting}

or imperatively with Javascript:

\begin{lstlisting}[language=JavaScript,numbers=none]
 var MyElement = document.registerElement('my-element');

 // instantiate a new instance of the element
 var thisOne = new MyElement();      
 document.body.appendChild(thisOne); // add to the <body>
\end{lstlisting}

With a simple example like this the result does not look all that different from a \tcode{<span>}.
To do something more interesting with your custom element you will need to the other features of Web Components: Shadow DOM, templates and imports.

\subsection{Shadow DOM}
Shadow DOM encapsulates the internal structure of an element. 
As we have seen, browsers use Shadow DOM to encapsulate the private state of standard elements like \tcode{<video>} but now this capability is extended to custom-defined elements.

TODO: 

\url{http://www.w3.org/TR/shadow-dom/}

You can think of Shadow DOM like an HTML fragment inside an element that describes its external appearance without exposing these structural details\footnote{
Shadow DOM should not be confused with the React framework's Virtual~DOM concept, which is closer in nature to HTML5 Templates than Shadow DOM.
}. Typically a custom element definition has a template (more on these in a moment) which produces the shadow DOM necessary to render the element.
The actual contents of the shadow DOM are just ordinary elements.

Custom elements can wrap regular text, normal HTML elements, or other custom elements and then project that content into its own internal structure.
In the example in figure~\ref{f:twbs2} above, a simple \tcode{<twbs-navbar>} element consumes a set of three \tcode{<a>} (anchor or link) elements
but internally transforms that to something like the example in figure~\ref{f:twbs1}, 
projecting the set of links into the nav menu structure with appropriate wrappers.

The \tcode{<content>} tag is used inside a custom element's template to indicate the spot where the consumed (wrapped) content should be \textit{projected}. 
This wrapped content is known as \textit{light DOM}, because it's given by the user and projected through into the shadow.
Together the shadow DOM and light DOM form the \textit{logical DOM} of a custom element.
It is also possible for elements to have multiple shadow DOM sub-trees. 
This is used particularly for emulating object-oriented-like inheritance relationships between custom elements.

In languages like C\# and Java, the encapsulation of classes and the protection of private object fields are a relatively strong guarantee by the language.
But in the case of Web Components, Shadow DOM is not completely and utterly isolated from the containing page.
It is possible to ``reach inside'' and break encapsulation to at least some degree, 
but the point is that this must be an intentional act by the developer and not an unexpected side-effect.

\subsection{HTML Imports}
One significant problem faced by web developers is the lack of any built-in packaging system for modules in HTML.
Prior to Web Components there was no way to import a snippet of HTML or Javascript from an external location and insert it exactly one time into the current document, similar to an \tcode{\#include} directive in the C language or the packaging and import systems popular in scripting languages like Python, Go and Ruby. 
Javascript could always be loaded with a \tcode{<script>} tag like usual, but this did not ensure that resources were loaded and executed exactly once, a process known as \textit{de-duping}.
A component that uses a certain JS resource might be found in two different spots on the page but that resource would be requested from the server twice, degrading application performance.

In order to fix these problems the HTML Imports standard allows for bringing in snippets of HTML, CSS or Javascript into the current document in a way that ensures automatic de-duping of repeated requests.
The one major caveat is that de-duping only happens if the resources are named in exactly the same fashion in each case.
Dealing with HTML Imports in a consistent fashion will be discussed in the Implementation section.

TODO: HTML imports:

\url{http://www.w3.org/TR/html-imports/}

\url{http://www.html5rocks.com/en/tutorials/webcomponents/imports/}


\subsection{Templates}
The last major piece of the Web Component puzzle is the native HTML5 \tcode{<template>} tag. 
Unlike the rest of Web Components, \tcode{<template>} has already become a standard part of the HTML5 specification, although one that is not yet widely used outside of WC.
\textit{Template} is a frequently overloaded word with different meanings in different programming environments.
While HTML5 templates have some similarities to the concept of templates popularized by frameworks like Angular and Django, there are some important differences.

TODO: Template elements:

\url{http://www.w3.org/TR/html5/scripting-1.html#the-template-element}

HTML5 templates are inert hunks of HTML embedded in the page that can be instantiated into `real' elements by Javascript.
Their basic function is to give a template for custom element representation.

However, templates are most useful in combination with `live' data, not static, unchanging text.
Binding data into templates with special operators\footnote{
Sometimes called mustaches, handlebars or curly braces. }
is \textbf{not} a part of the standard HTML5 template spec.
The following example of a data bound template is something that does \textit{not} work with plain Web Components alone:

\begin{lstlisting}[language=HTML5,numbers=none]
	<template> 
		The temperature is {{ temp }} in {{ city }} right now.
	</template>
\end{lstlisting}

Instead this functionality can be handled by a Javascript framework such as React or Polymer.
Data-bound templates are discussed in more detail in the following chapter.

The primary benefit of HTML Templates from a performance perspective is that external resources referenced from the template (images, stylesheets, etc) will not be fetched until the template is actually instantiated.
Templates are often used to declare the internal structure (shadow DOM) of custom elements. 
Therefore the resources needed to use the custom element aren't downloaded until they are actually needed, which is necessary when composing a large application out of numerous distinct elements.

% http://www.w3.org/TR/html5/scripting-1.html#the-template-element


\subsection{Related technologies}
There are a number of related W3C initiatives for web standards. 
Sometimes these are loosely grouped under the label Web Components,
and they do help support componentization, 
but they are separate part of the HTML5 standard.

Mutation observers:

\url{http://www.w3.org/TR/dom/#mutation-observers}

Model driven views:

\url{http://mdv.googlecode.com/svn/trunk/docs/design_intro.html}

Pointer events:

\url{http://www.w3.org/TR/pointerevents/}

Web animations:

\url{http://www.w3.org/TR/web-animations/}

Selectors  (similar to jQuery selectors)

\url{http://www.w3.org/TR/selectors-api/}

\begin{lstlisting}[language=JavaScript,numbers=none]
	// find an element based on 'id' attribute.
	var someElem = document.querySelector("#some-id");
\end{lstlisting}

\textit{Placeholder Notes}:
A significant problem with 'web components' story - scoping!  
There is just one global scope on the page.
This leads to the practice of 'prefixing as poor man's scope'.  
E.g. instead of facebook providing a <like-button> element, they provide <facebook-like-button>.
apparently this may be an area of future development  (citation?)

Flex boxes layout: intended for smaller page components...

\url{https://developer.mozilla.org/en-US/docs/Web/Guide/CSS/Flexible_boxes}


CSS Grid layout intended for overall page layout...

\url{http://www.w3.org/TR/css3-grid-layout/}


\section{Literature Review}
\subsection{Javascript frameworks}
\subsection{Polymer framework}
A team within Google has developed the Polymer~\cite{polymercontributors2015} framework based on Web Components architecture.

\section{Speakur}
My desire to learn more about modern web development led me to investigate web frameworks like Angular and Meteor.
I spent some time building (very) simple demos with these two frameworks in particular.
Although they are expressive and powerful, and are used every day to power high-traffic applications, 
I was unhappy with the non-standard and idiosyncratic nature of these frameworks. 
They relied on `proprietary' (even if open source) extensions that were not native to HTML and not easily transportable across different frameworks and architectures.
This dissatisfaction led me to learn about the Web Components initiative.

\subsection{Origin}
Learning about Web Components quickly led me to the Polymer project.
I wanted a component that demonstrated common use cases for Web Components and also showed off some of the design possibilities provided by Polymer and Material Design.
I was also intrigued by the possibilities of a server-free design afforded by Firebase.
Some kind of `live' social plugin seemed like a natural fit for the capabilities of Polymer and Firebase, so this led to a discussion plugin for blogs and other articles.
My hope was that it would required little or nothing in the way of dedicated server resources in order to actually use it. 

\subsection{Motivations}

I wanted my discussion component to have some of the following attributes:

\begin{itemize}
\item Provide a simple API to consumers that hid most implementation details.

\item Require minimal server resources. Ideally nothing would need to be ``installed'' and it could be loaded in a cross-origin fashion from online web developer tools like \url{https://jsbin.com}.

\item Support Markdown formatted comments including syntax highlighting for code snippets.

\item Support internationalization (i18n) and localization (l10n) features for a global audience.

\item Support distributed event notification similar to the publish-subscribe (pubsub) design pattern. 

In essence, `live' data updates:
when someone replies to a post it should become instantly available to anyone viewing the thread.

\item If any framework was used at all, it should be based on Web Components. 


This instantly ruled out the vast majority of frameworks, 
leaving only Polymer and the less-comprehensive X-Tags project~\cite{x-tagscontributors2015} and a few other smaller contenders.

\end{itemize}

In the next chapter we will discuss some of the high level architectural concerns that should be addressed when designing such a component.


\chapter{Approach and Software Architecture}
\index{Approach and Software Architecture%
@\emph{Approach and Software Architecture}}%
\label{ch:approach}

This chapter discusses the architectural approach behind Speakur, 
a social discussion plugin for the desktop and mobile web based on Web Components\index{Web Components} and the Polymer\index{Polymer} framework.
Web authors can use Speakur to easily add a comment section to their articles or blog posts.
Visitors can leave feedback about the article and engage in discussion with each other.
Discussions are grouped into topics or `threads', and within these threads, users can reply to the main article or to each other.

\section{Functionality and Features}
Speakur\index{Speakur|textbf} 
is a Custom Element\index{Custom Elements} 
(\tcode{<speakur-discussion>}\index{<speakur-discussion>|textbf}) 
that provides an embeddable discussion forum or comment hosting service for a blog, web page or other web application, similar to the commercial service Disqus\index{Disqus}\footnote{\url{https://disqus.com/}}.
Examples of Speakur's user interface\index{user interface (UI)} can be found in~\cref{f:demo1,f:lang}.
Placing Speakur inside a web page is straightforward.
As shown in~\cref{l:example1},
and in~\cref{publishing} below,
the web author simply places the \tcode{<speakur-discussion>}
 element in the HTML\index{HTML} at the desired spot.
This requires two supporting steps detailed in~\cref{publishing}:
\begin{enumerate}
\item loading the Web Components\index{Web Components} polyfill\index{polyfill} script
\item importing the \tcode{<speakur-discussion>} element.
\end{enumerate}

Once an author imports the element and places it on his page, that site now has an integrated discussion forum for desktop and mobile\index{mobile} users. 
All forum data including user profiles and comment text is stored in an online cloud database called Firebase~\cite{firebasecontributors2015}\index{Firebase}.
The messy details of structuring a discussion forum are abstracted\index{abstraction} away from the web page author.

The \tcode{<speakur-discussion>}\index{<speakur-discussion>} element presents a simplified application programming interface (API)\index{API|textbf} to its users.
There are only a few options available including the URL of the Firebase instance and the thread target URL (\tcode{href}\index{href}).
If you do not provide your own Firebase\index{Firebase} URL, by default, my demo database is used instead.
Therefore serious users will wish to use their own Firebase account with their own database and resource limits.

In addition to basic commenting features, Speakur offers the ability to vote comments up or down, custom profiles, 
the ability to leave comments in Markdown\index{Markdown} syntax~\cite{githubcontributors2015} with syntax highlighting\index{syntax highlighting} for common programming languages, 
and (rough) user interface translations  or localizations\index{localization}\index{internationalization} 
into 15 languages as shown in~\cref{f:lang}.

\begin{figure}[htb]
\centering
 \includegraphics[width=\textwidth]{images/screenshot_20150320_1923_lang.png}
\caption{Speakur's interface language\index{internationalization} updates instantly upon selection.}
\label{f:lang}
\end{figure}

From a technical perspective, one of the more interesting features in Speakur is how Polymer's\index{Polymer} data-bound templates\index{data-bound template}, also known as Model-Driven Views\index{Model-Driven View}, 
allow components to automatically reflect changes in widely separated (decoupled) areas of the application.
Also, Firebase's\index{Firebase} event notification\index{event notification} architecture allows all web clients to instantly and transparently reflect any changes in remote clients.
Another example of the power of data-bound templates is that all of the user-visible text in the application immediately updates as soon as the user changes his or her locale preference in the dialog in~\cref{f:lang}.

\section{Software Architecture Overview}
It has been said that ``any problem in computer science can be solved with another layer of indirection\index{indirection}'' (usually attributed to David Wheeler).
The key to understanding a software package is learning its architecture\index{architecture|textbf},
which is really a map of these layers of indirection\index{indirection} and abstraction\index{abstraction}.
Roy Fielding\index{Fielding, Roy}, the author of the influential REST\index{REST} web architecture, described a software architecture as:

\begin{quote}
\dots an abstraction of the run-time elements of a software system during some phase of its operation. A system may be composed of many levels of abstraction and many phases of operation, each with its own software architecture.

At the heart of software architecture is the principle of abstraction: hiding some of the details of a system through encapsulation\index{encapsulation} in order to better identify and sustain its properties. A complex system will contain many levels of abstraction, each with its own architecture~\cite{fielding2000}.
\end{quote}

The architecture for Speakur\index{Speakur} is based on client-side (browser) JavaScript\index{JavaScript} code and HTML\index{HTML} layout following the Web Components\index{Web Components} design principles listed in~\cref{sec:wcprinciples} below. 
There is no dedicated server component except for the Firebase\index{Firebase} cloud database service~\cite{firebasecontributors2015}.
Speakur is built entirely from plain HTML, JS and CSS\index{CSS} files that can be served from a content delivery network (CDN)\index{content delivery network (CDN)} 
such as \tcode{github.io}\index{GitHub} or your own server~\cite{landers2015-d}.
This low-overhead design allows Speakur\index{Speakur} to be used on your own website without actually `installing' any software;
just load Speakur directly from \tcode{github.io} with an HTML \textit{import}
and then insert a tiny bit of HTML markup into your document~\cite{landers2015-d}.
This helps fulfill the ease of use requirements 
\#\ref{motive:abstraction} and \#\ref{motive:cors}
in~\cref{motivations} on~\cpageref{motivations}.

\subsection{Speakur's Internal Components}
Most of the user interface\index{user interface (UI)} elements in Speakur\index{Speakur}---things like dropdown menus and dialogs as well as invisible functional components like expand/collapse elements---are implemented with Polymer's\index{Polymer} Core\index{Core Elements} and Paper\index{Paper Elements} custom element libraries.
Speakur presents a simple API through \tcode{<speakur-discussion>}\index{<speakur-discussion>},
but internally it consists of a number of internal abstraction layers or custom elements.
In turn, these internal layers consist of still more focused layers, other Polymer\index{Polymer} components, 
simple HTML templates, 
and wrappers around external JavaScript\index{JavaScript} libraries 
like \tcode{moment.js}\index{moment.js library}.
These internal layers are described in~\cref{sec:layout} on~\cpageref{sec:layout}.

As previously mentioned, in order to make the 
\tcode{<speakur-discussion>}\index{<speakur-discussion>} element available for use, 
you must first load the Web Components polyfill\index{polyfill} and then \textit{import} the Speakur element, 
either from \tcode{github.io} or your own server.
I recommend that you create your own Firebase\index{Firebase} instance for data security and resource limitation reasons, but even this is not required. 
If you don't create your own database, my demonstration instance is used instead.
Security\index{security} for web clients engaging in data manipulation 
(i.e., posting, editing or deleting comments) 
is handled entirely through the Firebase\index{Firebase} authentication\index{authentication} 
and data security\index{security} rules described below.
The simplicity of this arrangement makes it easier to fit Speakur\index{Speakur} into almost any web application architecture\index{architecture}.

\section{Responsive Design}
\label{bg:mobile}
Although native apps are frequently preferred by mobile users and developers, 
the mobile web remains an important development platform for the same reason that 
desktop web apps continue to live alongside native desktop apps; 
web apps typically do not require installing anything to the device, are always up-to-date, and have a lower barrier to entry for users and developers.
Because Speakur\index{Speakur} is not a standalone application but rather a plug-in designed to be embedded into other web pages or apps, 
the full document is not under its control.
This can affect the mobile\index{mobile} user experience\index{user experience (UX)},
but within these limits, Speakur strives to present a responsive\index{responsive design} interface to different screen sizes.

\begin{figure}[htb]
\centering
 \includegraphics[width=3in]{images/mobile2.png}
\caption{Speakur thread on a mobile phone.}
\label{f:mobile1}
\end{figure}

The two main techniques used in Speakur\index{Speakur} for responsive design are CSS\index{CSS} flexible (flex)\index{Flex} boxes~\cite{mozillacontributors2015} 
and CSS \tcode{@media}\index{@media} rules that apply varied styles to different screen sizes. 
In addition, Speakur's JavaScript\index{JavaScript} code exposes a \tcode{smallScreen} flag to the HTML templates\index{HTML!templates} that can trigger layout adjustments, 
such as moving the user photo to a different location.

The desktop version in~\cref{f:demo1} (\cpageref{f:demo1}) shows the user photo to the left of the main post area, 
but the mobile version in~\cref{f:mobile1} above shows how the user photo is moved to the right side of the post header. 
A CSS\index{CSS} \tcode{@media}\index{@media} rule for small screens disables the indentation of replies shown at the bottom of~\cref{f:demo1}. 
The extensive use of CSS flex\index{Flex} box attributes throughout the app ensures that structural elements like headers and toolbars automatically adjust their layout to available space~\cite{polymercontributors2015-d}. 
This is described in more detail in~\cref{sec:responsive} on~\cpageref{sec:responsive}.

\section{Polymer and Web Components}
As described in~\cref{ch:background}, 
Web Components\index{Web Components} are a 
W3C\index{W3C} initiative 
to expose certain native browser features in a public, standardized way. 
Polymer\index{Polymer} is a 
Google\index{Google} 
web framework built from the ground up around Web Components.
Polymer also provides the crucial 
polyfill\index{polyfill} library 
that is required for Web Component features to work on most current browsers.

In general, Speakur tries to adhere to the core principles laid out by the 
Web Components\index{Web Components} developers 
for general purpose components. 
It's worth quoting those here in full, 
because they are applicable to software engineering\index{software engineering} in general. They are:\label{sec:wcprinciples}
\begin{quote}
\begin{enumerate}
\item Address a common need.\label{wcp:commonneed}
\item Do one job really well.\label{wcp:onejob}
\item Work predictably in a wide variety of circumstances.\label{wcp:predicatable}
\item Be useful right out of the box.\label{wcp:useful}
\item Be composable.\label{wcp:composable}
\item Be styleable.\label{wcp:stylable}
\item Be extensible.\label{wcp:extensible}
\item Think small.\label{wcp:thinksmall}
\item Adapt to the user and device.\label{wcp:adaptable}
\item Deliver the key benefit to HTML authors, not just coders~\label{wcp:htmlauthors}\cite{webcomponentscontributors2014}.
\end{enumerate}
\end{quote}
% Explain how I follow those guidelines... [TODO]
% Will be explained in ch 4 and 5, should I mention that here? probably not

The Polymer\index{Polymer} project provides two (optional) libraries of Custom Elements\index{Custom Elements} for use in your own projects. 
\textbf{Core Elements}\index{Core Elements} includes both user interface\index{user interface (UI)}
widgets and invisible functional elements.
Core Elements are minimally styled and can be used directly, 
but they are also used as base classes for 
\textbf{Paper Elements}\index{Paper Elements}, 
which implement the so-called `Material Design'\index{Material Design} look and feel (design language) used by apps for Google's\index{Google} Android\index{Android} mobile\index{mobile} operating system~\cite{imura2015}.
The use of Paper Elements helps give web apps like Speakur a look-and-feel that resembles a native mobile app.
Speakur's\index{Speakur} layout is composed of a combination of native HTML5 elements, 
Core and Paper Elements, 
and Speakur's own custom elements which abstract\index{abstraction} internal implementation details.

\section{Firebase and Data Synchronization}
All persistent data in Speakur is stored in a cloud database\index{database} called Firebase~\cite{firebasecontributors2015}\index{Firebase|textbf}.
Anyone who wants to use Speakur\index{Speakur} can register for a free account on \tcode{firebase.com} and create a database instance to hold Speakur data.
No other server component is required.
Firebase is a NoSQL\index{NoSQL}-style key-value data store of the kind that has been popularized
with the growth of 
Node.js\index{Node.js}, a server-side JavaScript\index{JavaScript} environment,
and the MongoDB\index{MongoDB} NoSQL\index{NoSQL} database\index{database}~\cite{dickey2014}.

Firebase\index{Firebase|textbf} provides WebSockets\index{WebSockets}-based event notification\index{event notification} and synchronization 
as well as a security rule description format based on 
JavaScript Object Notation (JSON)\index{JSON|textbf}
%\footnote{JavaScript Object Notation (JSON)\index{JSON} is a popular data format because it's easily human-readable, 
%but its free-form structure compared to XML\index{XML} can be both a blessing and a curse.  
%It mirrors the syntax for writing data structure literals in the JavaScript\index{JavaScript} language.} 
for securing\index{security} and validating user actions.
The use of Firebase as the only server component allows for easy deployment\index{deployment} of Speakur with minimal dependencies, helping to adhere to the principles of ``think small'' and ``be useful right out of the box'' from the previous section.
%Although Speakur's Firebase library uses the WebSockets protocol for data transfer and network event notification, 
%Firebase also provides a RESTful programming interface that uses the standard HTTP protocol.
This also makes Speakur completely dependent on Firebase security rules for authorizing user behavior like deleting posts.

\subsection{RESTful API}
Firebase\index{Firebase|textbf} can be used by itself as the sole provider of data services to an application, as Speakur does, or else it can be used as an auxiliary to other services or REST\index{REST|textbf} APIs\index{API}.
One of the key architectural benefits of using Firebase, 
besides its ease of deployment, 
is that its data binding and event notification\index{event notification} system allows for 
applications to respond to changes in real-time while maintaining adequate performance.

Firebase\index{Firebase|textbf} itself provides a REST API\index{API} for data access by programs like Speakur.
The term REST\index{REST|textbf} or RESTful is sometimes misunderstood,
but in Roy Fielding's\index{Fielding, Roy} original 2000 Ph.D. thesis, 
REST refers to transferring \textit{representations} of application state and using hypertext as the engine of 
application state\index{HATEOAS}\footnote{Known under the somewhat awkward acronym of HATEOAS\index{HATEOAS}.}~\cite{fielding2000}.
Specifically, `objects' of whatever type are represented as interlinked hypertext \textbf{\textit{resources}} that are operated on by standard HTTP verbs\index{HTTP!verbs} such as \tcode{PUT}\index{HTTP!PUT} and \tcode{DELETE}\index{HTTP!DELETE}.
For example, in a Speakur\index{Speakur} thread, a particular user comment (a post) is an abstract \textit{resource} located at the following uniform resource locator (URL):

\tcode{https://YOUR-DB.firebaseio.com/posts/\$ParentId/\$PostId}

%\tcode{/posts/\$ParentId/\$PostId}

where \tcode{\$PostId} is an identifier (id) for the post itself, and \tcode{\$ParentId} is the id of the post this is a reply to, 
which could be the top-level post (thread id.) 
In order to delete this post, one would issue an HTTP\index{HTTP} request to this URL with the \tcode{DELETE}\index{HTTP!DELETE} verb, as in this \tcode{bash} shell script snippet:

\begin{lstlisting}[language=bash,caption=
{Deleting a post with the REST API.},label=l:rest_delete,captionpos=below]
set DATABASE="https://my-firebase.firebaseio.com"
set POST="fake_post_id"  # the post to delete
set PARENT="parent_post_or_thread_id"

curl --request DELETE $DATABASE/posts/$PARENT/$POST
\end{lstlisting}
\index{curl}\index{HTTP!DELETE}

Of course, the server would be expected to check your authorization to delete this resource.
Authenticating on the command line is not shown here for simplicity.
Viewing (reading) the post is done with a simple \tcode{GET}\index{HTTP!GET} request.
Creating a brand new post can be done with \tcode{PUT}\index{HTTP!PUT}.
Updating an existing post's text is done with the \tcode{POST}\index{HTTP!POST} verb.
By default, the resources are \textit{represented} as JSON\index{JSON} encoded data, 
but a client can request an alternative representation such as XML\index{XML}.
The storage format for the resource `at rest' inside the server database is irrelevant from the API\index{API} perspective.

Areas where typical web APIs\index{API} fall short of being truly ``REST-ful''\index{REST|textbf} include:
\begin{enumerate}
\item Treating URLs as endpoints for remote procedure calls (RPC\index{RPC}) instead of hypertext resources that link to each other in exactly the same way a website starts from the home page and links to various resources which in turn link to other resources.
\item Not closely following HTTP semantics, especially using HTTP verbs\index{HTTP!verbs} inappropriately like using \tcode{POST}\index{HTTP!POST} for all actions including deletion.
\item Not using content-related HTTP headers\index{HTTP!headers} appropriately for data representations and API\index{API} versioning\index{versioning}~\cite{steveklabnik2011}.
\end{enumerate}

The Firebase\index{Firebase|textbf} API\index{API} follows typical RESTful\index{REST|textbf} patterns in data access, allowing the database\index{database} 
and its metadata\index{metadata} to be addressed as a set of linked HTTP resources starting from a single root.
In addition, Firebase client libraries use a WebSockets\index{WebSockets} connection, 
a lower-level TCP/IP\index{TCP/IP} protocol, 
to perform event notification\index{event notification|textbf} and distributed synchronization without the overhead of polling or high-overhead HTTP 1.1\index{HTTP} requests.
WebSockets are used to enhance performance but all of the data in Firebase\index{Firebase|textbf} 
(and hence, in Speakur\index{Speakur}) 
is accessible from the RESTful\index{REST|textbf} HTTP API\index{API} outside of a WebSocket.

\subsection{WebSockets}
\label{sec:websockets}
The WebSockets\index{WebSockets|textbf} protocol, formally known as RFC 6455\index{RFC 6455}, 
is a TCP/IP\index{TCP/IP} protocol that can be used alongside HTTP\index{HTTP} for persistent data connections between web clients and servers~\cite{mozillacontributors2015-a}.
Its primary purpose is to avoid the overhead of having the client initiate a new HTTP connection to check on the status of something on the server, also known as polling\index{polling}.
Aside from the fact that an HTTP\index{HTTP} connection can be `upgraded' (protocol switched) to WebSockets,
there is no direct connection or dependency between HTTP and WebSockets.
They can be used independently.

Firebase\index{Firebase|textbf} uses WebSockets rather than traditional high-overhead HTTP\index{HTTP} requests to move data back and forth to the client.
This always-on connection allows for sending nearly instant event notifications\index{event notification} to all currently active clients with minimal overhead.
In practice, this allows the application to update its state in real-time as different users read and write values the database.
The Firebase\index{Firebase|textbf} client library `subscribes' to an area of interest in the database, such as the replies to a particular thread.
It then receives notifications when these areas change and so it can update the local representation as appropriate.
This library also allows authors to register error handler functions to be called in case something goes wrong, but as of this writing, these handlers are not being called due to an internal bug in the \tcode{<firebase-element>}\index{<firebase-element>} wrapper.

\section{Security Architecture}
\label{sec:arch_security}
Because Speakur\index{Speakur} relies entirely on Firebase\index{Firebase|textbf} for data persistence, its security\index{security|textbf} model is tied heavily to Firebase capabilities. 
All Speakur code runs inside the client's web browser including the small `admin' mode.
The security architecture\index{architecture} of the web is for the most part about protecting the \textit{user} from malicious servers (and other users), not protecting the \textit{server} from the user.
It is assumed that servers protect themselves from unauthorized actions.
Because Speakur\index{Speakur} has no `server' as such, 
other than the Firebase cloud service,
that means the security mechanisms that do things like prevent users from deleting each other's posts 
must be implemented entirely within Firebase security rules.
This is discussed in more detail in~\cref{sec:security}.

Firebase implements the two major categories of access control: authentication\index{authentication|textbf} and 
authorization\index{authorization|textbf}. 
Authentication (sometimes abbreviated \textit{authn}) answers the question ``who are you?'' 
while authorization (\textit{authz}) asks ``what are you allowed to see and do?''~\cite{stallings2011}.
Confidentiality of in-transit data is handled with Transport Layer Security (TLS)\index{transport layer security (TLS)}, 
better known as the secure socket layer (SSL)\index{secure socket layer (SSL)} or HTTPS\index{HTTP!HTTPS}.

\subsection{Authentication}
Speakur's authentication\index{authentication|textbf} and sign-in system is handled through Firebase\index{Firebase|textbf}, 
specifically Google\index{Google} and Facebook\index{Facebook} OAuth\index{OAuth} single-sign on (SSO)\index{single sign-on (SSO)}.
Users can sign into a Speakur discussion thread, and hence Firebase, through their Facebook or Google identity.
Firebase\index{Firebase|textbf} supports other authorization schemes, including account registration (``simple password''), Twitter\index{Twitter} and GitHub\index{GitHub} identities.
Site owners who use Speakur can also designate certain threads to allow anonymous commenting.

Every user who registers within a Speakur\index{Speakur} instance by signing in with one of those identity providers gets a unique identifier---the \tcode{uid} or user ID. 
This \tcode{uid} is used extensively in the database to refer to the user, including in the security (authorization\index{authorization}) rules that are external to the actual database (i.e., security metadata\index{metadata}.) 
Thread owners also have the option of allowing anonymous posts by users who are not signed in.

\subsection{Authorization}
Firebase\index{Firebase|textbf} lets developers write security\index{security|textbf} authorization\index{authorization} rules determine what level of access a user has within the system and what they are allowed to do or see.
Speakur's\index{Speakur} security\index{security|textbf} rules are described in more detail in~\cref{sec:security}.
The set of security rules for a Speakur database lives in a JSON\index{JSON} encoded file that contains expressions that determined whether a given database read or write (change) is allowed. 
This file resides on the Firebase server, which evaluates the rules when authorizing actions.
For example, I defined rules for Speakur which prohibit anyone from editing or deleting an existing post unless they are the original post author or an authorized administrator as shown here:

\begin{changemargin}{-0.3cm}{-0.8cm}
\begin{lstlisting}[language=JavaScript,caption=
{Security rules for the \tcode{posts} table (user messages).},label=l:sec_rule1,captionpos=below]
"posts": {
  ".read": true, // anyone can read posts
  ".write": false,  // deny modification by default
  "$parentId": {
    "$childId": {
      // must be admin or post owner to modify a post.
      ".write": "data.exists() ? ( auth.uid === data.child('author').child('uid').val() || root.child('admins').child(auth.uid).child('scope').val() === '*' ) : true",
      // validate structure of new posts
      ".validate": "newData.hasChildren([ 'threadId', 'text', 'author' ]) && newData.child('timestamp').val() > 1"
    }
  }
}
\end{lstlisting}
\end{changemargin}

The rule structure is dictated by the architecture\index{architecture} of Firebase\index{Firebase|textbf}.
This setup has certain pros and cons that can be seen in~\cref{l:sec_rule1}. 
On one hand, it's logically designed and the expressions provide a powerful and fairly comprehensive way to specify the 
security\index{security|textbf} and authorization\index{authorization} logic for your database.
On the other hand, non-trivial rule expressions can get awkwardly long and in complex cases can devolve into a maze of nested ternary operators.
Having one big file with all the JSON formatted security expressions needed for a database works great for simpler cases but may become unwieldy in a more complex application without the assistance of additional tools.

% Move the above to the analysis section?

\section{Data Flow and Event Handling}
One important area of Speakur's architecture is the flow of data within the system and responding asynchronously\index{asynchronous} to local and remote user events.
To understand the general flow of information in Speakur (and Polymer\index{Polymer}), imagine the component as an inverted tree, with the `root' at the top being the main \tcode{<speakur-discussion>}\index{<speakur-discussion>} element, 
and the nodes and leaves under that being the internal parts such as the headers, the posts, and other user interface\index{user interface (UI)} controls as shown in~\cref{f:component_layout}.
In this model, information flows \textit{down} through data bindings and bubbles \textit{up} through events\index{event notification}.

Data binding\index{data-bound template|textbf} is one of the most important extensions to standard DOM provided by the Polymer\index{Polymer} framework. 
This ties (or `binds') a variable from the data model to one or more spots in the shadow DOM, or else to an input or output from some other component, 
so that any change in the variable is reflected in the bound locations.
Two primary uses of data bindings are
binding a custom element's representation (shadow DOM template) to live data, 
and sending data to other elements or components~\cite{polymercontributors2015-b}. 
Data binding helps provide separation or decoupling between the user interface or \textit{view} and the underlying data \textit{model}, 
a design pattern that is commonly known as 
Model-View-Controller (MVC)\index{model-view-controller|textbf}.
A short example can be found in~\cref{l:dbtemplate} on~\cpageref{l:dbtemplate}.
\Cref{sec:databindings,sec:dbtemplates} discuss data binding and data-bound templates in more detail.
%  from~\cpageref{sec:databindings}

Although Polymer does support 2-way bindings\index{data-bound template}, 
experience has shown that when exchanging data with external components, 1-way (top-down) bindings are easier to reason about and should be preferred.
Bound variables should flow in one direction, generally parent to child as mentioned above.
So how should a child communicate changes back to the parent? 
The parent (or any other interested party) can register a DOM\index{DOM} mutation observer to be notified when the child value changes, or the child can fire an \textit{event}\index{event notification} that the parent can respond to.

As mentioned in~\cref{sec:bgmutation}, 
mutation observers\index{mutation observers} are a standard way to register a callback\index{callback} to be run when some area of the page (DOM) changes. 
A callback in this case is a function that reacts to a change in some other part of the system.
This doesn't necessarily have to be tied to a specific variable like with data-bound templates\index{data-bound template} and can reflect any change in any DOM attribute for standard \textit{and} custom elements. 

\section{Deploying Speakur}
Speakur\index{Speakur} is designed to be easy to deploy\index{deployment}.
Adding a discussion forum to a blog or web app is as simple as altering the page to:

\begin{enumerate}
\item load Polymer's\index{Polymer} Web Components\index{Web Components} polyfill\index{polyfill},
\item import the \tcode{<speakur-discussion>}\index{<speakur-discussion>} element,
\item then place \tcode{<speakur-discussion>}\index{<speakur-discussion>} on the page at the desired location. 
\end{enumerate}

Because steps 1 and 2 above can load these resources from an external service or content delivery network\index{content delivery network (CDN)},
Speakur does not require installing any software to the web server.
The HTML\index{HTML}, CSS\index{CSS}, and JavaScript\index{JavaScript} files, plus associated resources like images and JSON\index{JSON} language files, 
can all be loaded from a different web host. 
This is known as cross-origin resource sharing, or CORS\index{Cross-origin resource sharing (CORS)}.
Speakur\index{Speakur} avoids a common security restriction associated with CORS known as Same-Origin Policy\index{Same-Origin Policy}, which is used to prevent a class of attacks called Cross-site Request Forgery (CSRF)~\cite{mozillacontributors2015-b}.
It does this by relying entirely on Firebase for live data,
which in turn uses WebSockets\index{WebSockets},
which are not subject to Same-Origin Policy\index{Same-Origin Policy} restrictions on asynchronous JavaScript HTTP\index{HTTP} requests (also known as AJAX\index{AJAX} or \tcode{XMLHttpRequest}\index{XMLHttpRequest}).
This allows Speakur to be used on a site without that site having installed a copy of its files or directly serving them to clients.

\subsection{Software Dependencies}
\label{sec:dependencies}
Like most applications, Speakur\index{Speakur} relies heavily on other libraries and frameworks to implement some of its underlying behaviors. 
Some of these libraries in turn depend on other libraries.
Collectively, all of the third party modules required to run Speakur\index{Speakur} are called \textit{dependencies}\index{dependencies}.
For example, safe HTML rendering of user-supplied Markdown-style text is handled by the excellent Marked library\index{Marked library}~\cite{christopherjeffrey2014}.
See~\cref{appendix:credits} on~\cpageref{appendix:credits} for the full list of Speakur's\index{Speakur} direct dependencies.

Taking advantage of the de-duping\index{de-duping} feature of HTML Imports\index{HTML!Imports}
requires that you name the path to your resources and dependencies in a consistent fashion.
In other words, if Component A and Component B both require Component Z, both A and B must use the same URL when requesting (importing) Component Z.
Furthermore, each of Z's dependencies must also follow this same pattern when importing their own dependencies in order to fully realize de-duping\index{de-duping}.
Therefore Speakur follows the recommendation of the Polymer\index{Polymer} framework and uses Bower\index{Bower} to manage dependencies~\cite{bowercontributors2015}. 
Speakur only needs to list which packages it requires, either by name or their Git\index{Git} repository address, 
and Bower\index{Bower} handles downloading these to a managed component directory alongside Speakur itself.

\subsection{Vulcanize}
One possible `problem' with adopting Web Components\index{Web Components} architecture\index{architecture} is that it encourages the partition of a large application into numerous smaller components or files.
Of course, this is really a \textit{feature} designed to enhance code organization, encapsulation\index{encapsulation} and productivity, 
but it has one important side effect: 
under the HTTP\index{HTTP} 1.1 protocol that currently powers the web, each small component requires a separate request which degrades page loading times.
Techniques like \tcode{Keep-Alive} headers can help, and the HTTP 2.0\index{HTTP!HTTP 2.0} protocol addresses this question more comprehensively but is not yet finalized or widely adopted.

In the meantime, the Polymer\index{Polymer} project provides a tool called Vulcanize\index{Vulcanize|textbf} which concatenates and minifies\index{minify} your web components into a single file.
Concatenation puts all of the required text resources (HTML, CSS, and JS) into a 
single\footnote{Except when Content Security Policy (CSP)\index{content security policy (CSP)} requires the enforced separation of HTML, CSS, and JS. In this case, three files are used.}
file to reduce the number of HTTP\index{HTTP} requests.
Minification\index{minify} removes all comments, whitespace and non-essential elements from the code to compact it and reduce transfer times.
Whether using Vulcanize results in an overall faster site ultimately depends on a number of factors including how and when the components are used.
For that reason, empirical testing is needed to know whether it should be used in production on a real site~\cite{polymercontributors2015-a}.
\Cref{table:speakurloading} on \cpageref{table:speakurloading} illustrates how Vulcanize helps improve loading performance in three browsers.
Every time I release a new version of Speakur\index{Speakur}, I update the \tcode{gh-pages} branch in its Git\index{Git} repository with a Vulcanized edition that includes everything needed to run it, including Polymer\index{Polymer} and all of its other dependencies.
This publishes it to the \tcode{github.io} CDN and makes it available to users.


\chapter{Implementation}
\index{Implementation%
@\emph{Implementation}}%


\section{Layout and Structure}
\label{sec:layout}

Speakur\index{Speakur} is delivered as a single Custom Element\index{Custom Elements}, \tcode{<speakur-discussion>}\index{<speakur-discussion>}.
You must import the tag with \tcode{<link rel=import href=...>} tag to make it available to place on the page. 
This element has a data-bound\index{data-bound template} \tcode{<template>}\index{<template>} that provides the element's visual representation (shadow DOM).
Actually, a custom element's representation, or logical DOM\index{Logical DOM}, may consist of both encapsulated shadow DOM\index{Shadow DOM} as well as light DOM\index{Light DOM} that is supplied by the \textit{user} of the custom element and \textit{projected} into the shadow DOM as discussed in section~\ref{bg:shadowdom}.
In the case of Speakur, the discussion forum is self-contained in terms of content and light DOM is not needed or used.
Speakur's\index{Speakur} options or parameters are controlled with attributes\index{HTML!attributes}:

\begin{lstlisting}[language=HTML5,caption=
{Using HTML attributes to set Speakur options},label=l:options1,captionpos=below]
 <speakur-discussion
   firebaseLocation="https://speakur-demo.firebaseio.com"
   xtitle="This is the thread's (initial) title."
   href="demo1"
   initiallyOpen="true"
   allowAnonymous="true">
 </speakur-discussion>
\end{lstlisting}

Structurally, Speakur\index{Speakur} consists of JavaScript\index{JavaScript}, HTML, and CSS\index{CSS} files, along with a few other resource types like images and JSON\index{JSON} language files. 
All of the these internal resources and components are hidden from consumers who only have to import and place the main \tcode{<speakur-discussion>}\index{<speakur-discussion>} element as discussed in section~\ref{publishing} below.

\begin{table*}\centering
\ra{1.3}
\begin{tabular}{@{}lp{8cm}@{}}
\toprule
Name & Function \\
\midrule
$Structural~containers$\\
\tcode{<speakur-discussion>} & Top-level public component. \\
\tcode{<speakur-thread-view>} & Container for the comments displayed in an entire thread. \\
\tcode{<speakur-card>} & Provides a Material Design\index{Material Design} `card' container. \\
\tcode{<speakur-post-set>} & A container for a list of posts such as replies to a specific post. \\
\\
$Logical~services$\\
\tcode{<speakur-base>} & Base `class' for most other Speakur components. \\
\tcode{<speakur-i18next>} & Element wrapper around i18next library. \\
\tcode{<speakur-profile>} & User session data and preferences. \\
\tcode{<speakur-post-vote>} & Controller for voting posts up or down. \\
\\
$Interface$\\
\tcode{<speakur-compose>} & Composing replies and posts. \\
\tcode{<speakur-post>} & Displays a single user post/comment. \\
\tcode{<speakur-login-button>} & Login/logout button and dropdown. \\
\tcode{<speakur-theme>} & Base class for all themes, mainly CSS rules. \\
\tcode{<speakur-theme-blue>} & A specific theme. \\
\tcode{<speakur-dialog-profile>} & Dialog to edit user preferences. \\
\tcode{<speakur-lang-select>} & Dropdown to choose the UI language. \\
\bottomrule
\end{tabular}
\caption{A selection of Speakur's internal components.}
\label{table:speakurcomponents}
\end{table*}


Internally, Speakur\index{Speakur} is composed of a few broad categories of sub-components that are listed in table~\ref{table:speakurcomponents}. 

asdf



\section{Database Design}
\label{sec:database}

\section{Synchronization}
\label{sec:sync}

\subsection{Data Bindings}
\label{sec:databindings}

\subsection{Data-Bound Templates}

\section{Security}
\label{sec:security}

\subsection{Authentication}

\subsection{Firebase Security Policy Rules}
\label{sec:security}

\section{Internationalization}
\label{sec:i18n}

\subsection{Architecture}
\subsection{Libraries}

\section{Responsive Design}
\label{sec:responsive}

\subsection{Mobile Users}

\subsection{Accessibility}

\section{Publishing and Deployment}
\label{publishing}


\chapter{Analysis}
\index{Analysis%
@\emph{Analysis}}%

This report aims to demonstrate how certain traditionally difficult problems in web development have been addressed by Web Components\index{Web Components}.
The principles underlying Web Components and Polymer\index{Polymer} are summarized by the list in section~\ref{sec:wcprinciples} taken from ~\cite{webcomponentscontributors2014}.
I have tried to make Speakur\index{Speakur} adhere to these principles both internally and externally,
and to demonstrate the application of abstraction\index{abstraction} and encapsulation\index{encapsulation} patterns to creating flexible, composable, reusable, sharable components.
My goal is to understand how the W3C Web Component\index{Web Components} initiative will impact web application engineering\index{software engineering} in the future.
By all appearances, the long term impact will be substantial.

\section{Web Components architecture}
Web Component's\index{Web Components} essential benefit is that it allows decoupling of components, 
separation of concerns, 
and the encapsulation of internal structure.
This raises bundles of discrete functionality or `components' to a higher level of abstraction and 
will help open up a new ecosystem of reusable, extendable, sharable modules.
The Web Component principles from section~\ref{sec:wcprinciples} emphasize thinking small, 
breaking the problem down into manageable chunks and being adaptable, extendable and composable.
Web developers were already doing this in many cases, but without formal support from the browser.

Perhaps the most interesting piece of advice is to ``deliver the key benefit to HTML authors, not just coders''~\cite{webcomponentscontributors2014}.
It underscores that, fundamentally, 
the web consists of \textit{documents} with \textit{authors}, 
a fact that is sometimes forgotten by the myriad of JavaScript\index{JavaScript} libraries and frameworks\index{framework}.
The components that will find the most success will be those which focus first and foremost on providing convenience and benefit to HTML\index{HTML} authors and content developers, 
not just the software engineers\index{software engineering} who maintain the libraries and frameworks behind the scenes.
Speakur\index{Speakur} aims to provide as small a deployment surface as possible to web authors so they can easily add a discussion forum to their site without embedding a big blob of HTML.
The growth of social coding sites like GitHub\index{GitHub} will further encourage the sharing and reuse of Web Components designed along these principles.

\section{Writing Web Components with Polymer}
This section details some of the lessons learned developing Speakur\index{Speakur} on top of Google's\index{Google} Polymer\index{Polymer} framework.
Polymer's data bindings\index{data-bound template} are a powerful feature for driving `live' web applications.
Their utility extends well beyond user interface\index{user interface (UI)} views.
A 3-way automatic binding between the local model, the UI view, and the remote database is an extremely powerful abstraction,
even more so when combined with Firebase's\index{Firebase} 
Web\-Socket-based\index{WebSockets} event notification\index{event notification} system.
JavaScript variables can automatically update to reflect changes made by users on the other side of the world.
Of course, care must be taken to design the interface so the user doesn't accidentally make changes that are immediate and can't be undone. 
For example, Speakur's user preferences screen doesn't commit changes to the database until the user clicks the Save button.
Editing the text of an existing post also saves a copy of the previous version.

Data bindings and synchronization are described in section~\ref{sec:sync}.
In general, two-way data bindings should only be used \textit{within} a component.
Across component boundaries, use a one-way binding to send data down to a child,
and fire events to notify containers of state changes.
DOM Mutation Observers\index{mutation observers} can be useful for making one component react to a change in another without directly tying their code together.
As the example with \tcode{lc} (locale) showed in section~\ref{sec:i18n}, 
sometimes it's necessary to `shoehorn' certain variables into a Polymer binding expression 
to ensure that the expression is recalculated when that variable changes.

Component load times are currently one problem area for Web Components.
The Vulcanize\index{Vulcanize} tool offers definite benefits for load times as seen in Table~\ref{table:speakurloading},
but Speakur is a relatively simple application where all 20 or so components are used more or less immediately.
Vulcanization must be applied at the level of the site or app, not by the individual component authors.
Larger single page applications (SPAs) will face more complex challenges when trying to optimize load times, and performance in general,
before Web Components are fully native in browsers, especially when combining frameworks.

The \tcode{<template>}\index{<template>} element is a powerful new tool for web authors, 
but current implementations can have performance issues when rendering large lists of items, such as with a long list of comments. 
Polymer's \tcode{<core-list>} is more optimized for this use case.
Also, the \tcode{if=} attribute on nested \tcode{<template>}\index{<template>} elements should not be used in those cases where you could use the \tcode{hidden?=} attribute on the inner content instead.
There are several reasons for this, including performance and 
the fact that Polymer's handy \tcode{this.\$} \tcode{id}-to-object map is not dynamic; 
that is, element objects created conditionally in inner templates are not automatically added to \tcode{this.\$}.

I recommend avoiding the use of Polymer's\index{Polymer} inheritance unless there is a clear case for providing polymorphism-like behavior.
One reason for that is that (as of this writing) when a base class provides computed properties, 
these property definitions have to be repeated in any subclasses if any new ones are required,
which violates the ``don't repeat yourself'' (DRY)\index{don't repeat yourself (DRY)} principle.
Speakur uses Polymer inheritance in its CSS\index{CSS} theme components; 
most of the functionality is defined in the base class and the subclasses mainly provide numeric parameters for things like colors and margins.
Composition could achieve these behaviors almost as easily.

%Having multiple <core-overlay> active at the same time can cause problems. 
% I created a simple fix that adjusts the CSS \tcode{z-index} property. 
%You can find the fix here:~\cite{landers2015-d}.

\section{The future of Web Components}

Polymer\index{Polymer} and Web Components\index{Web Components} are extremely powerful tools.
They are relatively immature as a technology but quickly improving.
Those who wish to deploy a large, full-featured ``prod\-uction-quality'' web application based on these technologies may face some bumps in the road, 
at least until the standard is finalized and browser support is improved.
A project version below 1.0 is a general indicator of how ready its authors feel it is for production deployment, and Polymer's\index{Polymer} current version is at 0.5, soon to be 0.8, 
which suggests that 1.0 is not that far off.
Even the massively popular Angular\index{Angular} framework has announced that the controversial 2.0 re-write will be based on the Web Components standard,
so it appears that Web Components are here for the long haul~\cite{santiagoesteva2015}.

The Web Components\index{Web Components} initiative is much bigger than just the Polymer project.
The draft is due to be formally adopted as an official HTML5\index{HTML!HTML5} standard by the World Wide Web Consortium\index{W3C}.
Although it will still be a long time before full-speed native Web Component support is found in a majority of browsers worldwide, 
polyfill\index{polyfill} libraries can help bring these capabilities to current browsers.
However the prospect of forcing visitors to download a several hundred KB library just to use basic DOM functionality is not very appealing to most web engineers from a performance and user experience\index{user experience (UX)} perspective, 
so the lack of native browser support may be a limiting factor in Web Component adoption in the near term.

In essence, Web Component's\index{Web Components} key problem at the moment is that one has to download a large JavaScript\index{JavaScript} framework\index{framework} in order to use it, 
yet one of its main premises is that it frees you from needing a JavaScript framework.
Of course, the path out of this chicken-and-egg problem is for browser vendors to fully implement native Web Components in their products,
a process which won't be complete until well after the standard is finalized by the W3C.
Still, Speakur\index{Speakur} has demonstrated that it's possible to create focused, extensible, yet full-featured Web Components right now with the help of tools like Polymer\index{Polymer} and Firebase\index{Firebase}.

\chapter{Conclusion}
\index{Conclusions%
@\emph{Conclusions}}%

Web Components\index{Web Components} are a powerful addition to the web developer's arsenal. 
Their fundamental premise is that web content can be componentized, encapsulted\index{encapsulation}, and abstracted\index{abstraction} behind well-defined APIs\index{API}.
Eventually this functionality will be delivered directly by the browser without needing a dedicated JavaScript\index{JavaScript} framework.
%But for the immediate future, using Web Components in your application means that your clients must download a large blob of JavaScript library code -- the polyfill\index{polyfill} --- 
%that enables Custom Elements\index{Custom Elements}, 
%shadow DOM\index{Shadow DOM}, 
%templates\index{HTML!Templates}, and 
%the component import system\index{HTML!Imports}.
But for the immediate future, using Web Components in your application means that your clients must download a large blob of JavaScript library code---the polyfill\index{polyfill}---to enable its features: 
Custom Elements\index{Custom Elements}, 
shadow DOM\index{Shadow DOM}, 
templates\index{HTML!Templates}, and 
the component import system\index{HTML!Imports}.

Web Components have set out on the right path to becoming a mature, widely deployed technology. 
Newer frameworks like Polymer\index{Polymer} provide powerful abstractions but are still undergoing rapid evolution.
Older, more established frameworks like Angular\index{Angular} have already begun the process of adapting to the new Web Component world. 
Success will be found at the end of this road as long as framework and component authors remain mindful of the guiding principles of Web Components: 
to be small, focused, extensible, adaptive, 
and most of all, 
to deliver the key benefits 
not only to the software engineers\index{software engineering} who make the web run,
but to the content \textit{authors} that have made it the indispensable communications medium of the modern age.



%%%%%%%%%%%%%%%%%%%%%%%%%%%%%%%%%%%%%%%%%%%%%%%%%%%%%%%%%%%%%%%%%%%%%%
% Appendix/Appendices                                                %
%%%%%%%%%%%%%%%%%%%%%%%%%%%%%%%%%%%%%%%%%%%%%%%%%%%%%%%%%%%%%%%%%%%%%%
%
% If you have only one appendix, use the command \appendix instead
% of \appendices.
%
%\appendices


% http://tex.stackexchange.com/a/103173/75161
%\setcounter{table}{0}
%\setcounter{page}{1}
%\newcommand{\initAnhang}{
%    \renewcommand{\thepage}{A\arabic{page}}
%    \newpage
%}
%\renewcommand\appendix{\par
%        \renewcommand\thesection{A}
%        \renewcommand\thesubsection{A\arabic{subsection}}
%        \renewcommand\thetable{A\arabic{table}}}        
%\appendix\initAnhang


\appendix
\index{Appendices@\emph{Appendices}}%

%\chapter{Speakur source code example}
\index{Appendix!Speakur source code@\emph{Speakur source code}}%

This appendix may contain an example of Speakur's source code...




%%%%%%%%%%%%%%%%%%%%%%%%%%%%%%%%%%%%%%%%%%
\chapter{Open Source Credits}
\index{Appendix!Open Source credits@\emph{Open Source credits}}%
\label{appendix:credits}

Speakur would not have been possible without the following open source components.
The components listed here are the direct Bower\index{Bower} dependencies.
Some of these may install other dependencies in turn.

\begin{itemize}
\item Polymer\index{Polymer} framework~\cite{polymercontributors2015}: 

Web Components\index{Web Components} polyfill\index{polyfill}, core library, Core Elements\index{Core Elements} and Paper Elements\index{Paper Elements}.
\item Firebase\index{Firebase} database service and client library~\cite{firebasecontributors2015}
\item Marked\index{Marked library} library~\cite{christopherjeffrey2014}
\item highlight.js\index{highlight.js library} library
\item i18next\index{i18next library} library~\cite{i18nextcontributors2015}
\item moment.js\index{moment.js library} library
\item lodash.js\index{lodash.js library} library
\item jQuery.js\index{jQuery} library
\item js-md5 library\index{js-md5 library}
\item pvc-globals\index{<pvc-globals>} custom element
\end{itemize}


% \include{chapter-appendix3}

\printindex{}

%%%%%%%%%%%%%%%%%%%%%%%%%%%%%%%%%%%%%%%%%%%%%%%%%%%%%%%%%%%%%%%%%%%%%%
% Generate the bibliography.					     %
%%%%%%%%%%%%%%%%%%%%%%%%%%%%%%%%%%%%%%%%%%%%%%%%%%%%%%%%%%%%%%%%%%%%%%
%								     %
% NOTE: For master's theses and reports, NOTHING is permitted to     %
%	come between the bibliography and the vita. The command      %
%	to generate the index (if used) MUST be moved to before      %
%	this section.						     %
%								     %
\nocite{*}      % This command causes all items in the 		     %
                % bibliographic database to be added to 	     %
                % the bibliography, even if they are not 	     %
                % explicitly cited in the text. 		     %
		%						     %

\begingroup
%\raggedright
%\sloppy
%\renewcommand*{\bibfont}{\raggedright}
\setlength{\bibitemsep}{6pt}
\printbibliography[heading=bibintoc]
\endgroup

%\bibliographystyle{plain}  % Here the bibliography 		     %
%\bibliography{diss}        % is inserted.			     %
\index{Bibliography@\emph{Bibliography}}%			     %
%%%%%%%%%%%%%%%%%%%%%%%%%%%%%%%%%%%%%%%%%%%%%%%%%%%%%%%%%%%%%%%%%%%%%%


%%%%%%%%%%%%%%%%%%%%%%%%%%%%%%%%%%%%%%%%%%%%%%%%%%%%%%%%%%%%%%%%%%%%%%
% Generate the index.						     %
%%%%%%%%%%%%%%%%%%%%%%%%%%%%%%%%%%%%%%%%%%%%%%%%%%%%%%%%%%%%%%%%%%%%%%
%								     %
% NOTE: For master's theses and reports, NOTHING is permitted to     %
%	come between the bibliography and the vita. This section     %
%	to generate the index (if used) MUST be moved to before      %
%	the bibliography section.				     %
%								     %
% \printindex%    % Include the index here. Comment out this line      %
%		% with a percent sign if you do not want an index.   %
%%%%%%%%%%%%%%%%%%%%%%%%%%%%%%%%%%%%%%%%%%%%%%%%%%%%%%%%%%%%%%%%%%%%%%


%%%%%%%%%%%%%%%%%%%%%%%%%%%%%%%%%%%%%%%%%%%%%%%%%%%%%%%%%%%%%%%%%%%%%%
% Vita page.							     %
%%%%%%%%%%%%%%%%%%%%%%%%%%%%%%%%%%%%%%%%%%%%%%%%%%%%%%%%%%%%%%%%%%%%%%

\begin{vita}
\renewcommand{\thefootnote}{\fnsymbol{footnote}}
Preston Brent Landers
was born in north Texas and attended high school on the Nevada side of Lake Tahoe. 
He received his Bachelor of Arts in English from the University of Texas at Austin with a minor in Government.
He works as a web and mobile software engineer for Journyx, Inc.\footnote[1]{\url{http://www.journyx.com}}
in Austin, Texas and began graduate studies in Software Engineering at 
the University of Texas at Austin in August 2012.

\end{vita}

\end{document}
